\documentclass[11pt]{article}

\usepackage{graphicx}
\usepackage[hidelinks=true]{hyperref}
\usepackage{appendix}

\usepackage[numbers]{natbib}

\title{Practical Network Defense notes}
\author{Matteo Salvino}
\date{}

\begin{document}
\maketitle
\pagebreak
\tableofcontents
\pagebreak
\section{Network 101}
Internet is in essence a network of networks with a hierarchical structure. The path into the Internet backbone could be wired or wireless. The backbone itself consists of global Internet Service Providers (ISPs) and several regional ISPs that are all interconnected to provide a path from sender to receiver. The communication path may typically contain a variety of switches and routers that facilitate and direct the flow of information through the network. The communication links, regardless of whether they are wired or wireless, are defined by a transmission rate and bandwidth. Access networks are used to connect a host or Local Area Network (LAN) to the Internet. Routers connect  LANs, generate routing tables and forward  packets of data on their path from source to destination. The Internet backbone, also called network core, is basically a group of routers interconnected by optical fiber as well as DNS servers containing infrastructure name servers, such as root Domain Name Servers (DNSs) employed for naming. The remaining components in the Internet structure that lie outside the network core, are simply access networks. An individual home network can be considered a small network or subnet. The Internet uses a gateway, also known as an edge router, to connect these networks to the edge of the network, also called network edge. The network core is composed of a set of routers and fiber links. The routers work together to determine the most efficient routing path for a packet from source to destination. A distributed algorithm is used that provides the flexibility to adapt to changing conditions, and routing tables are generated and maintained in real time. The core is provided by ISPs that interconnect multiple continents, called Global ISPs or Tier-1 ISPs, whereas the Regional ISPs are known as Tier-2 ISPs. The tier-1 ISPs that form the Internet backbone are interconnected at various access points called \textbf{Internet eXchange Points (IXPs)}. It would certainly appear that the intercommunication among computers would require some standardization that would facilitate their successful interactions. There should be some "protocol" that defines the manner in which they talk to each other so that messages are clearly understood. It is this "protocol", documented in a stack that is accomplished through modularization, development and upgrades that support operations such as web surfing, email, etc. By its very structure it is clear that the stack consists of different layers, each  of which performs a special function. Modularization of the Internet is accomplished through layering. As a result, the Internet is being developed by many people, and institutions through a divide-and-conquer strategy. There is a strong interaction between layers in that each layer relies on the services of the layer below and exports services to the layer above. It's the interface between layers that defines the interaction, for example implementation details can be hidden and layers can change without affecting other layers. Access layer is constituted by network 
with the same end-point, and the protocol used is Ethernet. Each host in a Ethernet network has a Network Interface Card (NIC) with a fixed address. MAC addresses uniquely identify hosts in the network and each of them processes  packets intended for it. An Ethernet network constitutes a broadcast domain, where each Ethernet frame is received  by all hosts. Actually, switches segment the network to limit the explosion of packets in the network, and only  broadcast message are replicated. Switches remembers the source MAC addresses on the different ports. They only replicate the frame on the segment where the destination MAC address replies, with the support of ARP tables (for hosts) and CAM tables (for switches). Ethernet cannot be used for the entire Internet because of its broadcast packets, that would be highly inefficient for large networks. There is the need of a logical division of the networks : while Ethernet provides the Access layer, the Distribution layer is based on \textbf{Internet Protocol (IP)}, which is at level of Autonomous Systems (like big enterprises and ISPs). The main difference between Ethernet and IP address is that
the first one is a physical addresses, so we cannot change the MAC address of our NICs, and tell who you are but not where you are, the second one is a logical address, so we can change the IP address of our NIC and is used to identify and reach networks and hosts. If two hosts belongs to the same network, then they can communicate using
local addresses, otherwise they must use remote addresses. In order to know if one IP is in the same network than you, we can use the subnet mask. There are two version of IP addresses :
\begin{itemize}
\item \textbf{IPv4} : it defines IP address with 32 bits organized in four octets (8 bits in each)
\item \textbf{IPv6} : has 128 bits.
\end{itemize}
For human readability the bits in each octet are separated by dots while writing an IPv4 address and colons in IPv6. IP addresses, regardless its version, are constituted by a network part and an host part, where the subnet mask defines precisely the boundary between them. There are three types of IP addresses :
\begin{itemize}
\item \textbf{Unicast} : these refer to a single destination host
\item \textbf{Broadcast} : these refer to every host on a network or subnet
\item \textbf{Multicast} : these refer to a group of IP addresses in a network, not necessarily all of them.
\end{itemize}
IP addresses has been classified in the following classes :
\begin{itemize}
\item Class A (24 bits for host addresses)
\item Class B (16 bits for host addresses)
\item Class C (8 bits for host addresses)
\item Class D (multicast)
\item Class E (reserved).
\end{itemize}
There are routable and non-routable address ranges. The first one need to be unique on the Internet (public), while the second ones (private, e.g. from 10.0.0.0 to 10.255.255.255, from 172.16.0.0 to 172.31.255.255, from 192.168.0.0
to 192.168.255.255) are defined in RFC 1918. Since each set of 8-bits can hold values from 0-255 we haven't flexibility at all. The idea is to use a Variable Length Subnet Mask in order to use a specific length of the subnet mask based on our needs. Many smaller networks can be grouped using the supernetting technique. In point-to-point links the subnet 31 is allowed. Other ways to reduce the waste of IP addresses in a subnet are NAT and IPv6. To conclude this section we want to link a useful reference to deeply understand how networks really works \cite{introduction}.
\section{Traffic monitoring}
It's important to note that activities within the Internet can be approached in a modular fashion and this modularization is accomplished through layering. It is clear that the stack consists of different layers, each of which performs a special function. There is a strong interaction between layers in that each layer relies on the services of the layer below and exports services to the layer above. Encapsulation is a technique that allows us to use a layered architecture in communication. In fact, the data are transferred between application layer of two devices, where each layer introduce some protocol information and provides data to the layer below, until reaching the physical one. This latter sends the data over a physical medium such as copper, wired or fiber. On the destination side, the physical layer sends the data up to the stack, where each protocol will read the appropriate protocol information and forwards the data to the layer above, until reaching the application one. The task of a layer involves the exchange of messages that follow a set of rules defined by a protocol. When computers are connected with a network, guidelines must be established that support their interaction. The architecture that defines the network functionality is split into layers that collectively form what is commonly known as a protocol stack. Lets talk about \textbf{OSI model}. Each layer of the protocol stack may employ several protocols that implements the functionality of that particular layer. In a natural progression up the stack, the physical layer deals with the transmission of bits that are propagating over such media as copper, fiber or radio. The data link layer aggregates the bits (e.g. in a frame), and performs the data transfer between neighboring network elements using as an example, Ethernet or WiFi. The network layer handles the routing of datagrams, in packet form, from source to destination using routing protocols. The transport layer performs the process-to-process communication using segments, i.e. message transfer using for example TCP. The session layer aggregates connections for efficiency, synchronization and recovery in data exchange. The presentation layer permits applications to deal with coding, encryption, and so on. Finally, we have the application layer which allow the user to invoke protocols for information exchange. Another well known model is \textbf{TCP/IP}. It's constituted by four layers : application layer  corresponds to the top three layers of the OSI model, transport layer is equivalent to the transport layer of the OSI model, Internet layer is equivalent to the network layer of the OSI model, Datalink is equivalent to the data link layer of the OSI model and physical layer is equivalent to the physical layer of the OSI model. The layer ideal representation is that transport represent the illusion of direct end-to-end connection between processes in arbitrary systems, network transfers data between arbitrary nodes and data link transfers data between directly connected systems (via shared medium or direct cable). Each layer in the stack, with the exception of the physical layer, has a \textbf{header}. These headers facilitate the communication of information and are analogous to an envelope that contains both source and destination addresses. The link layer has a header containing MAC addresses, the network layer has a header containing IP addresses and the transport layer has a header containing the port  number (i.e. service number). The port range is [0..65535], where the source port is randomly chosen by the OS and the destination port determines the required service. The port range [0 ..1023] contains well-known ports and they are used by servers for standard internet applications (e.g. 25 for SMTP, 80 for HTTP, 143 for IMAP). The ports in the range [1024..49151] can be registered with Internet Application Naming Authority (IANA), whereas ports in the range [49152..65535] are called ephemeral ports. Protocols are classified in two main categories :
\begin{itemize}
\item \textbf{connection oriented} (like TCP) : these protocols perform a number of very important functions. For example, they govern the movement of packets from source to destination under the specifications of certain standards, take actions that are specified in the packets, manage packet flow and congestion for optimal performance and even recover lost packets. The protocols work in conjunction with one another to accomplish the specified task required by the user.
\item \textbf{connectionless} (like UDP) : these protocols doesn't provide any guarantee that the datagram is really deliver to the correct destination (i.e. there isn't control on data exchange). Furthermore, there isn't a possibility to recover from errors and manage the flow congestion.
\end{itemize}
The procedure to open a TCP connection between two hosts, A and B, is called three-way handshake, which is constituted by the following steps :
\begin{enumerate}
\item A sends a SYN segment to B with the sequence number field that contains the value x, which specifies the initial sequence number of A.
\item B sends a SYN and ACK segment to A, where the sequence number field contains the value y, which specifies the initial sequence number of B and the ack field contains the received value x increased by one.
\item A sends an ACK segment to B, where the sequence number field contains the value x + 1 and the ack field contains the received value y increased by one.
\end{enumerate}
Packets that flow in the network can be captured using a network traffic dump tool, like dumpcap, wireshark or tcpdump. All of them can visualize and store the captured data, while the last two can also analyze the captured packets. In wireshark data from a network interface are "dissected" in frames, segments and packets, understanding where they begin and end. Then, they are interpreted and visualized in the context of the recognized protocol. When we captures packets, we collect a lot of them, so to make our life easier we can use filters. In fact, they allows us to only focus on requested packets or certain activity by network devices. Filters are divided in : display filters in order to inspect only the packets we want to analyze once the data has been processed, and capture filters in order to limit the amount of network data that goes into processing and is getting saved. In the second one the packets not captured are lost, where in display filters we display only captured packets matching the filter, without discard them. Frames are collected from the interface and passed to several, consecutive, "dissectors" one for each layer (they are passed from the bottom layer to the upper one). Protocols can be detected in two ways :
\begin{itemize}
\item \textbf{Directly} : if a frame has the field that explicitly states which protocol it is encapsulating.
\item \textbf{Indirectly} : with tables of protocol/port combinations and heuristics. 
\end{itemize} 
We can capture network traffic in different modes :
\begin{itemize}
\item \textbf{promiscuous mode} : cannot fetch every packet due to network segmentation.
\item \textbf{physical tap} : device that mirrors traffic passing between two network nodes.
\item \textbf{port mirroring} : a port of the switch which mirror all the traffic.
\item \textbf{aggressive approaches} like :
\begin{itemize}
\item \textbf{ARP poisoning} : send ARP replies to steal IP addresses
\item \textbf{MAC flooding} : fill the CAM table to make switch act like an hub.
\item \textbf{DHCP redirection} : exhausts IP addresses of the pool and then pretends to be the default gateway of the network with the new DHCP requests.
\item \textbf{ICMP redirection} : use ICMP type 5 message to indicate a better route.
\end{itemize}
\end{itemize}
Now, the question is How to prevent packet capture ? We can use two different approaches :
\begin{itemize}
\item \textbf{Dynamic address inspection} : validates ARP packets with IP-to-MAC inspection by intercepting every ARP request and dropping the invalid ones (placed on switches).
\item \textbf{DHCP snooping} : always implemented in switches, it distinguishes between trusted and untrusted ports and uses a database of IP-to-MAC. Ports that show rogue activity can also be automatically placed in a disabled state.
\end{itemize}
\section{IPv6 introduction}
The information about this section have been taken from Graziani's book \cite{ipv6_fundamentals}. IPv4 provides for a maximum 4.29 billion 32-bit addresses, which seemed like more than enough addresses when IPv4 became a standard in 1980. The number of IP addresses needed today far exceeds the world's population for several reasons. First of all, IPv4 addresses are often allocated in groups of addresses, as network addresses. Places such as companies, schools and airports are allocated network addresses
for their users. But a much larger reason we need so many more IP addresses is the number of devices per person that are being connected to the Internet. IPv6 provides more addresses than IPv4. In comparison, the 128 bit IPv6 address space provides for 340 undecillion addresses (i.e. approximately $2^{128}$). The number of people accessing the Internet is increasing dramatically. Even with short-term solutions like NAT, we are in the final stages of public IPv4 address availability. There has also been extraordinary growth in the number of devices connected to the Internet. IANA assigns IPv4 addresses to the five Regional Internet Registries (RIRs) in /8 address blocks. Using this address space, RIRs then redistribute the IPv4 network addresses to ISPs and other end customers. On January 31, 2011, IANA allocated the last two blocks of IPv4 address space to RIRs. At this point, IANA had now run out of IPv4 addresses. This doesn't mean that IPv4 addresses are no longer available for end customers, but
they can still get them from most ISPs. However, many ISPs are severely limited. So, in the early 1990s, IETF began the development of a new version of IP known as \textbf{IP Next Generation}, which later became \textbf{IPv6}. However, IPv6 was a long-term solution, and a short-term solution was needed immediately. Several short-term solutions were put into place. Two of the most important were \textbf{CIDR} and \textbf{NAT} with private IPv4 addresses. We know how CIDR allocates IP addresses. NAT along with private IPv4 addresses has been the reason IPv4 has survived all these years. It allows multiple hosts using private IPv4  addresses within their internal network to share one or more public IPv4 addresses when accessing the global Internet. NAT has also some critical points. At the very least, NAT means that our routers,
application gateway and other devices must perform extra processing to make NAT work, which also causes latency. Other points are that it breaks peer-to-peer networking and accessing our "hidden" system from other networks. The IETF never intended NAT to be used for security. We want to say that in the late 1970s, a family of experimental protocols, known as \textbf{Internet Stream Protocol (ST)}, and later \textbf{ST2}, was developed. ST was an experimental resource reservation protocol intended to provide Quality of Service (QoS) for real-time multimedia applications such as video and voice. Although it was never recognized as \textbf{IPv5}, when encapsulated in IPv4, ST uses IP version 5. 
\subsection{Notation and types}
IPv6 is much more than just a larger source and destination IP address. Developers of IPv6 took this opportunity to not only improve IP but also many of the protocols and processes related to IP. In fact, the benefits on IPv6 include : larger address space, stateless autoconfiguration, end-to-end reachability without private addresses and NAT, better mobility support, peer-to-peer networking easier to create and maintain, and services like VoIP and QoS become more robust. An IPv6 address is 128 bits in length and hexadecimal is the ideal number system for representing long strings of bits. Every 4 bits can be represented by a single hexadecimal  digit, for a total of 32 hexadecimal values from 0000 to FFFF. AS described in RFC 4291, the preferred form to represent an IPv6 address is x:x:x:x:x:x:x:x. Each x is a 16-bit section that can be represented using up to four hexadecimal digits, with the sections (also called \textbf{hextets}) separated by colons. The result is eight 16-bit sections for a total of 128 bits in the address. There exists some rules in order to reduce the notation involved in the preferred format such as :
\begin{enumerate}
\item \textbf{omit leading 0s} : one way to shorten IPv6 addresses is to omit leading 0s in any hextet. This rule applies only to leading 0s and not to trailing 0s (because omitting both would cause the address to be ambiguous).
\item \textbf{omit all 0s hextets} : the second rule for shortening IPv6 addresses is that you can use a double  colon (::) to represent any single, continuous string of two or more hextets consisting of all 0s. If there are multiple possible reductions, RFC 5952 states that the longest string of zeroes must be replace with double colon and if they are equal then only the first string of 0s should use the :: representation.
\end{enumerate}
An IPv6 address can be \textbf{unicast}, \textbf{multicast} and \textbf{anycast}. A unicast address uniquely identifies an interface on an IPv6 device. A packet sent to a unicast address is received by the interface that is assigned to that address. Similar to IPv4, a source IPv6 addresses must be a unicast address. In the following sections we will deeply describe the previous types of IPv6 addresses.
\subsubsection{Unicast addresses}
An unicast IPv6 address in turn can be :
\begin{itemize}
\item \textbf{Global Unicast Address} : it's a globally routable and reachable address in the IPv6 Internet. They are equivalent to public IPv4 addresses. Its generic structure is dictated by the \textbf{3-1-4 rule} : 
\begin{itemize}
\item $3$ means that the first $3$ hextets are used for the \textbf{Global Routing Prefix}. It's the prefix or network portion of the address assigned by the provider to the customer site.
\item $1$ means that the next hextet is used for the \textbf{Subnet ID}. It's a separate field for allocating subnets with the customer site. Unlike with IPv4, it's not necessary to borrow bits from the Interface id (host portion) to create subnets. The number of bits in the subnet id falls between where the GRP ends and where the interface id begins.
\item $4$ means that the last four hextets are used for the\textbf{Interface ID}. It identifies the interface on the subnet, equivalent to the host portion of an IPv4 address. The interface id in most cases is 64 bits.
\end{itemize}
This rule is very effective because allows us to specify different subnets only changing one hexadecimal digit in the subnet id. Except under very specific circumstances, all end users will have a global unicast address (or more than one). It's range is from 2000::/64 to 3fff:fff:fff:fff::/64.
\item \textbf{Link-Local Unicast Address} : a link-local address is an unicast address that is confined to a single link, a single subnet. They needs to be unique on the link and don't needs to be unique beyond the link. Therefore, routers do not forward packets with a link-local address. Link-Local Unicast  addresses are in the range of fe80::/10 to febf::/10. The following 54 bits are recommended to be all 0s and 64 bits dedicated to the Interface id. They are fundamental in IPv6 because a device can use a link-local address before it gets one dynamically. The devices uses the RA's source address as the default gateway, and this address is the router link-local one. Another benefit is that allows router to exchange routing messages.They are created in two ways :
\begin{itemize}
\item \textbf{Automatically} : an IPv6-enabled device must have a link-local address. By default, devices automatically create their own link-local unicast addresses. The prefix is typically fe80::/64, followed by a 64 bit interface id that is automatically generated in one of the two ways :
\begin{itemize}
\item \textbf{EUI-64 generated} : the IEEE defined the EUI process using the interface's Ethernet MAC address to generate a 64-bit interface id. When EUI-64 is used to create a link-local address, the fe80::/64 prefix is prepended to EUI-64 generate interface id. An Ethernet MAC address is 48  bits, a combination of a 24 bit Organizationally Unique identifier (OUI) and a 24 bit Device Identifier, written in hexadecimal. The EUI process is an insertion of 16-bit value of fffe between the 24 bit OUI and the 24 bit device identifier, with the Universal/Local bit flipped (7th bit of the first byte).       
\item \textbf{Randomly generated} : EUI-64 is a convenient technique for automatically creating a 64 bit interface id from a 48 bit MAC address. However, it introduces a concern for some users : the ability to trace an IPv6 address to the actual physical device using the 48 bit MAC address. To alleviate this privacy concern, devices can use randomly generated 64 bit interface ids.
\end{itemize}
\item \textbf{Statically} : the disadvantage of an automatically generated link-local address is that the long interface ID is difficult to recognize. It's much easier to use a simpler, manually configured link-local address that is easier to identify. 
\end{itemize}
\end{itemize}
\subsubsection{Multicast addresses}
Multicast is a technique in which a device sends a single packet to multiple destinations simultaneously (one-to-many), in contrast to a unicast address, which sends a single packet to a single destination (one-to-one). An IPv6 multicast address has the prefix ff00::/8, which defines a group of devices known as a multicast group. It's the IPv4 equivalent of 224.0.0.0/4. A packet sent to a multicast group always has a unicast source address. A multicast address can only be a destination address and can never be a source address. The typical structure of a multicast address is the following :
\begin{itemize}
\item ff00::/8 : the first 8 bits are 1-bits(ff), reserved for IPv6 multicast.
\item Flags : the next 4 bits are allocated for flags. The fourth flag is the transient flag, which denotes two types of multicast addresses :
\begin{itemize}
\item \textbf{Permanent (0)} : these addresses are assigned by IANA and include both well-known and solicited-node multicast.
\item \textbf{Non permanent (1)} : these are transient or dynamically assigned, multicast addresses. They are assigned by multicast applications. 
\end{itemize}
\item Scope : the scope field defines the range to which routers can forward the multicast packet (0 reserved, 1 interface-local scope, 2 link-local scope, etc.).
\item Group id : the next 112 bits represent the group id.
\end{itemize}
There are two types of multicast addresses :
\begin{itemize}
\item \textbf{Assigned} : RFC 2375 defines the initial assignment of IPv6 multicast addresses that have permanently assigned Global IDs. IANA maintains the list of well-known IPv6 multicast addresses in a registry called \textbf{IPv6 Multicast Address Space Registry}. For example when the group id value is one, then the multicast packet is sent to all network devices, when is 2 to all routers in the network, 5 to all OSPF router, and so on.
\item \textbf{Solicited-Node} : a solicited-node multicast address is automatically created and assigned to an interface for every global unicast address, unique local address, and link-local address on that interface. These multicast addresses are automatically generated using a special mapping of the device's unicast address with the solicited-node multicast prefix ff02:0:0:0:0:1:ff00::/104.
\end{itemize}
One of the benefits of using a layer 3 multicast address is that it is mapped to a layer 2 Ethernet MAC address. This allows the frame to be filtered by the switch. This means these packets will only be forwarded out ports where there are devices that are members of the multicast group.
\subsection{Addressing}
Using \textbf{ICMPv6 Neighbor Discovery Protocol} Router Solicitation and Router Advertisement messages, hosts determine how to obtain their IPv6 addressing information dynamically. If SLAAC is used, the host can automatically obtain IPv6 addressing information, such as prefix, prefix length, and a default gateway address. The default gateway address is a link-local address, the IPv6 source address of the Router Advertisement message. ICMPv6 ND includes similar processes as in IPv4,  such as address resolution, router discovery, and redirect, but also with some significant differences like prefix discovery, Duplicate Address Detection (DAD) and Neighbor Unreachability Detection (NUD). Neighbor Discovery uses five ICMPv6 messages :
\begin{itemize}
\item \textbf{Router Solicitation} message  |
\item \textbf{Router Advertisement} message | for router-device messages used with dynamic address allocation
\item \textbf{Neighbor Solicitation} message  |
\item \textbf{Neighbor Advertisement} message | for device-device messages used for address resolution
\item \textbf{Redirect} message | for router-device messages used for better first-hop selection.  
\end{itemize}  
A host sends a Router Solicitation message when it needs to know how to dynamically obtain its addressing information. This typically occurs during startup and is the default on most host operating systems. A Cisco router sends Router Advertisement messages every 200 seconds by default, and it also sends a RA message upon receiving a Router Solicitation message from a device. The RA message is a suggestion to devices on the link about how to obtain their addressing information dynamically. The RA
message is sent to all IPv6 devices with a multicast address. The RA message contains three flag :
\begin{itemize}
\item \textbf{Address Autoconfiguration} flag (A flag) : when set to 1, which is the default setting, this flag tells the receiving host how to use SLAAC to create its global unicast address. SLAAC allows the host to create its own GUA by combining the prefix in the RA message with a self-generated interface id (using EUI or randomly generated).
\item \textbf{SLAAC with Stateless DHCPv6} flag (O flag) : when set to 1 (default setting is 0), this flag tells the host to obtain other addressing information, other than its GUA, from a stateless DHCPv6 server. This information may include DNS servers addresses and a domain names. 
\item \textbf{Managed Address Configuration} flag (M flag) : when set to 1 (default setting is 0), this flag tells a host how to use a stateful DHCPv6 server for its GUA and all other addressing information. This is similar to DHCP for IPv4. The only information the host uses from the RA message is the RA's source IPv6 address, which it's used as the default gateway address.
\end{itemize}
In the second case after a device generates one or more addresses using SLAAC, it contact a stateless DHCPv6 server for additional information. Remember that a stateless DHCPv6 server doesn't allocate or maintain any IPv6 global unicast addressing information. A stateless server only provides common network information that is available to all devices on the network, such as a list of DNS server addresses or a domain name. In particular, a host after obtaining its own GUA sends out a DHCPv6 SOLICIT message to all DHCPv6 servers multicast addresses. One or more DHCPv6 server respond with a DHCPv6 ADVERTISE message, indicating that they are available for DHCPv6 service. The host responds to the selected server by sending an information REQUEST message, asking for other configuration  information. The selected DHCPv6 server responds with a REPLY message that contains the other configuration information.\\\\In the third case, stateful DHCPv6 doesn't uses SLAAC to generate a GUA. A stateful DHCPv6 server provides IPv6 GUAs to clients and keeps track of which devices have been allocated which IPv6 addresses. A significant difference between stateful DHCPv6 and DHCPv4 is the advertising of the default gateway address. In IPv4, the DHCPv4 server usually provides the default gateway. In IPv6, only the router transmitting the ICMPv6 Router Advertisement can provide the address of the default gateway dynamically. There is no option within DHCPv6 to provide a default gateway address. A host after receiving the RA uses the source IPv6 address of the RA as its default gateway address. Then addressing and other configuration information is available from a stateful DHCPv6 server. So, the host sends out a DHCPv6
SOLICIT message to all DHCPv6 servers multicast addresses, searching for a DHCPv6 service. One or more DHCPv6 server respond with a DHCPv6 ADVERTISE message, indicating that they are available for DHCPv6 service. The host select one DHCPv6 server by sending to it a DHCPV6 REQUEST message asking for addressing and other configuration information. The selected DHCPv6 server responds with a reply message that contains a GUA and other configuration information. Finally, the host perform DAD on the address received from the stateful DHCPv6 server to ensure that this address is unique.\\\\ To ensure unicast address uniqueness, address resolution procedure includes \textbf{Duplicate Address Detection (DAD)}. The device sends a Neighbor Solicitation message for its own IPv6 address to detect whether another device on the link is using the same address. If a Neighbor Advertisement message is not received, the device knows its address is unique on the link. 
\subsubsection{Prefix Delegation process}
In the world of IPv4, most internal networks use a private IPv4 address space for internal devices and NAT at the edge router to translate an address to a globally routable public IPv4 address. This is a common mechanism for most home networks and is partly responsible for keeping IPv4 alive for so many years. Avoiding the complications and problems with address translation, IPv6 uses a different technique. One of the methods IPv6 uses is DHCPv6 with the \textbf{Prefix Delegation} option, which provides a mechanism for automated delegation of globally routable IPv6 prefix from a provider's router to a customer's router using DHCPv6. In this process are involved two routers :
\begin{itemize}
\item \textbf{Requesting Router (RR)} : this is the router that acts as the DHCPv6 client, requesting the prefix(es) to be assigned.
\item \textbf{Delegating Router (DR)} : this is the router that acts as the DHCPv6 server, responding to the  requesting router's IPV6 prefix request.
\end{itemize}
Lets describe in details the Prefix Delegation process. First the RR's ISP facing interface needs an IPv6 address, that can be dynamically obtained using SLAAC, stateless or stateful DHCPv6 server. Then the RR initiates DHCPv6-PD in its SOLICIT message by including a request for an IPv6 prefix. The REPLY message from the DR (the ISP router) includes the IPv6 prefix. This is the prefix that the RR can use for its own internal network (i.e. a prefix that the RR can use to allocate addresses to its clients).
\subsection{Comparing the IPv4 and IPv6 headers}
In comparing the two headers, notice that the IPv6 header takes advantage of 64-bit CPUs, respect to the 32-bit CPUs of the IPv4 header. As a result, all IPv6 fields start on an even 64-bit boundary or a multiple of 64. The advantage is that 64-bit CPUs can read one 64-bit-wide memory word at time. However, this structure doesn't negatively affect 32-bit CPUs because a 64-bit boundary is also a 32-bit boundary. A quick comparison of the two headers reveals that the IPv6 header is a simpler protocol with fewer fields than its IPv4 counterpart. This makes IPv6 a leaner protocol and provides more efficient processing. Both IPv4 and IPv6 begin with the \textbf{Version} field, which contains the version number of the IP header. The IPv4 \textbf{Internet Header Length (IHL)} field, is the length of the IPv4 header in 32-bit words, including any optional fields or padding (i.e. IPv4 has a variable header length). IPv6 doesn't have an IHL field because the main IPv6 header has a fixed length of 40 bytes, which allows for more efficient processing. The IPv4 \textbf{Type of Service (ToS)} field and the IPv6 \textbf{Traffic class} field are identical fields. In particular, they are used to specify what type of treatment the packets should receive from routers. This information helps to provide QoS features by offering different degrees of precedence. The IPv6 \textbf{Flow Label} field, is a new field used to tag a sequence of flow of IPv6 packets sent from a source to one or more destination nodes. This field can be used by a source to label sequences of packets for which it requests special handling by the IPv6 routers, such as "real-time" service. The IPv4 \textbf{Total Length} field is the length of the entire IPv4 packet, measured in bytes, including the IPv4 header and the data. The IPv6 \textbf{Payload Length} field is a 16-bit field that indicates the length in bytes of just the payload following the main IPv6 header. It does not include the main IPv6 header, but only the payload length and extension headers length. The design of IPv4 accommodates MTU differences by allowing routers to fragment IPv4 packets when an MTU along the path is smaller than the sender's MTU. If a router receives an IPv4 packet that is larger than the MTU of the outgoing interface, this packet can be fragmented, depending on the options in the IPv4 header. Sometimes packets are fragmented into multiple packets at the source. The final destination of the IPv4 packet is responsible for reassembling the fragments into the original full-size IPv4 packet. Fragmentation divides IPv4 packets so that they can be forwarded out a link that doesn't support the size of the original packet. IPv4 \textbf{Identification, Flags and Fragment Offset} fields are used for packet fragmentation and reassembly. Unlike in IPv4, an IPv6 router doesn't fragment a packet unless it's the source of the packet. Intermediate nodes do not perform fragmentation. When an IPv6 router receives a packet larger than the MTU of the output interface, the router drops the packet and sends an ICMPv6 Packet Too Big message back to the source. This message includes the MTU size of the link in bytes so that the source can change the size of the packet for retransmission. A device can use \textbf{Path MTU Discovery} to determine the minimum link MTU along the path. The IPv4 \textbf{Protocol} field indicates the protocol carried in the data portion of the IPv4 packet. IPv6 has a similar field, the \textbf{Next Header} field, that specifies the type of header expected after the main IPv6 header. The same values used in the IPv4 Protocol field are used in the IPv6 Next Header field, along with some additional values for IPv6. Some of the most common values for both protocols are 6 for TCP and 17 for UDP. The IPv4 \textbf{Time To Live (TTL)} and IPv6 \textbf{Hop Limit} fields ensure that packets do not transit between networks for an indefinite period of time, as in the case of a routing loop. These fields are decremented by one each time a router receives the IPv4 or IPv6 packet. When the field contains the value zero, the packet is discarded, an ICMPv4 or ICMPv6 Time Exceeded is sent to the source of the packet. A \textbf{checksum} field for the IPv4 header is provided for protection against any corruption in transit. Each router along the path verifies and recomputes this field. If the checksum fails, the router discards the packet. There is no checksum field in the IPv6 header. The designers of IPv6 didn't include this field because layer 2 data link technologies such as Ethernet perform their own checksum and error control. Because there is no checksum field in IPv6, the UDP checksum is mandatory in IPv6. The most notable and recognizable difference in IPv6 compared to IPv4 is the expansion of the source and destination address from a 32-bit address in IPv4 to a 128-bit address in IPv6. The IPv4 \textbf{Option} field is optional. It's variable in size and rarely used, so it isn't usually included in most IPv4 packets. The IPv4 \textbf{Padding} field is used only if the IPv4 Options field is used and the size of the IPv4 header is no longer a multiple of 32 bits. When this is the case, the IPv4 header is padded with bits that are 0s so that it ends on a 32-bit boundary. Extension headers are an important addition to IPv6. They are optional headers that add flexibility and allow for future enhancements to IPv6. The main IPv6 header include a Next Header filed, which has one of two purposes : to identify the protocol carried in the data portion of the packet or to identify the presence of an extension header. The next header field indicates an additional header known as extension header. Immediately following the mandatory main IPv6 header, there can be zero, one or several extension header. It also allow the main IPv6 header to be a fixed size for more efficient processing. IPv4 uses options and padding fields to accomplish what the extension headers do in IPv6. The next header field is used to chain together multiple IPv6 headers and the data portion of the packet at the end of the chain. RFC 2460 recommends that when multiple extension headers are used in the same packet, those headers should appear in the following order : 
\begin{enumerate}
\item Main IPv6 header
\item Hop-by-Hop options header
\item Destination option header
\item Routing option header
\item Fragment header
\item Authentication Header (AH)
\item Encapsulating Security Payload (ESP) header
\item Destination Options header
\item Upper-layer protocol header.
\end{enumerate}
\subsection{Link-local attacks}
In this section we want to explain in details some common types of attacks over link-local addresses.
\paragraph{Network eavesdropping} There is somebody within the network that capture packets transmitted by others and read the data content in search of sensitive information (such as password and session tokens). A possible solution to avoid data eavesdropping is to use encryption. In fact, using a strong encryption algorithm, the attacker will be not able to decrypt sensitive information. It's important not only to protect the content of the packets, but also some information that are used to move the packets (in this way we enforce privacy measure), using for example \textbf{TOR} or a \textbf{VPN}. Network eavesdropping is done by using tools called \textbf{network sniffers}, which goal is to focus on really breaking the confidentiality of the data. In particular, they analyze the collected data, filters all the valuable information and provide them at user level. They don't generate any kind of traffic, i.e. work in passive mode. To successful perform an eavesdropping attack, you need to be along the path of the communication. To realize this type of attack, first of all your network interface should be placed in promiscuous mode, i.e. you need to remove from the drivers of your network card the check of your MAC address, in order to accept every network packets. In wireless LAN the things are even worst, in fact you can potentially receive not only the packet of your network but also the packets of other networks that have a station that is in the range of your wireless card. So, we can place a sniffer in several places like :
\begin{itemize}
\item \textbf{Non-switched LAN (LAN with HUBs)} : it's the ideal case, because the hub duplicates every frame to all the ports.
\item \textbf{LAN with switches} : in the switch the whole network is segmented, so we try to alter some of the mechanism of the switch, trying to make it to send the packets also in our segment (see MAC flooding). Or we can also use an ARP spoof attack, convincing the source of the packet that to send the packet to its intended destination, must send the packet to us. Lets see in details these two attacks :
\begin{itemize}
\item \textbf{MAC flooding} : before considering the switches there were the bridges. The bridges were the first devices that were able to connect two network segments. The idea is to extends the broadcast domain but also reducing the collisions. These devices uses the \textbf{store \& forward} technique, with which read and regenerate a frame only if needed. If we imagine that the bridge can have multiple port (i.e. a switch), then the mechanism is exactly the same, analyze the packet and understand on which port is connect a segment where the desired destination is connected. Whether we generate a packet we insert our MAC address in the packet as source MAC address and the switch builds this association between a port, a segment and a list of the hosts in that segment considering the MAC source address. The MAC address is saved in the \textbf{CAM Table}, using the content itself to understand where to save that information in memory. What happens when a switch receive a packet in case of destination MAC unknown ? It's clear that it doesn't known on which port is the segment that host the destination MAC, so it replicates the packets on all the ports (flooding on the network). This behavior is used for the \textbf{CAM overflow} attack, where the CAM table is filled with useless MAC addresses and then whenever we have no more space to store a MAC address, a cache miss, we have to replicates the packets. Usually switches use hash to place MAC in CAM table,i.e. like hashed lists, where buckets can keep a limited number of values; if the value is the same there are n buckets to place CAM entries, if all n are filled the packet is flooded. In order to avoid this type of attack, we can employ the port security in switches :
\begin{itemize}
\item allows us to specify MAC addresses for each port, or to learn a certain number of MAC addresses per port.
\item upon detection of an invalid MAC the switch can be configured to block only the offending MAC or just shut down the port.
\end{itemize}
\item \textbf{ARP spoofing} : An ARP request message should be placed in a frame and broadcast to all computer on the network. Each computer receives the request and examines the IP address. The computer mentioned in the request sends a response, while all other computers process and discard the request without sending any response. We need ARP to resolve an IP address to MAC address. For example, host A wants to communicate with host B. Host A realize that host B is in the same network, so they can communicate directly using Ethernet. So, we need a MAC address of B but A don't know it. Host A sends an ARP request, including the IP address that he's looking for, to all network hosts. Subsequently, host B recognize its own IP address and will answer sending back an ARP-reply to host A. The ARP reply has the MAC address of B as source and MAC address of A as destination, picked from the original ARP request. Then host A can finally sends directly the IP packets to host B. The main problem with this protocol is that it's untrusted, i.e. we cannot really trust the information that you see in ARP request and ARP reply. Whenever we receive an ARP reply, we store this information in an \textbf{ARP table}, which is a temporary table that is filled with this IP-MAC pairings. Whether we want to send a packet to a host in the same network, we check if the ARP table contains the MAC address corresponding to the target IP address. In a positive case, we use exactly that MAC address, otherwise we proceed as before (ARP request - ARP reply). Another mechanism used by this table is the \textbf{address aging}, which remove unused MAC addresses after a specific timeout. Whether we receive an ARP reply from the network, we extract the IP-MAC pairing and update the corresponding entry in the ARP table. This activity is typically performed with packets that are called \textbf{Gratuitous ARP}. It's an ARP reply that is not really an answer to an ARP request, but it's something performed by ourselves, used to announce our association between IP and MAC address. In this way, we avoid duplicate IP addresses in the network. What if we misuse the Gratuitous ARP ? A situation in which a host frequently announce its pairing using not his IP address but another one, for example the address of the network default gateway. So, all the other network hosts will be fooled to think that the default gateway is no more the MAC of the real router, but now the new MAC is the one of the attacker. In this case, the attacker will receive all the packets that are intended to go outside the network. What we have is an hijacking in the local network, i.e. you can pretend to be anybody in your network. The very first type of attack is the \textbf{Denial of Service}. Another type is the \textbf{Man In The Middle} attack, in which the attacker intercept the traffic and reroute it to get the reply, then forward the reply back.
\end{itemize}
\item \textbf{Wireless LAN} : if we used weak encryption or no encryption, the scenario becomes equivalent to LAN with HUBs. If we use more recent encryption tools like WPA then we are protected, i.e. all the transmissions that we can potentially capture, are encrypted.
\end{itemize}
\paragraph{IPv6 Neighbor Discovery threats} ICMPv6 Redirect can be abused in some way to create a rogue router. Whenever we have a host that sends a packet to the router because it has to be sent to a network X and the router reaches another router in the same network, then the first router forward the traffic to the second one, but sends also an ICMPv6 Redirect message to the host, informing he that he can reach network X with one hop less. In IPv6 a router can known if two hosts are not really in the same network, but if they are in the same link local. Some types of attacks based on IPv6 ND are :
\begin{itemize}
\item Neighbor Solicitation / Advertisement spoofing : same as ARP spoofing, i.e. pretends to be somebody else.
\item DAD DoS attack : imagine you use SLAAC in order to get a Global Unicast address, and the attacker replies to you that the address is used. This scenario is repeated and repeated, generating a DoS attack. At some point the host stop to sends Neighbor Solicitation, deciding to not be part of the network.
\item spoofed Redirect message, default router compromise, malicious last hop router (router threat).
\end{itemize}
\paragraph{Rogue Router Advertisement} What if the RA comes from a malicious router ? One of the most common problem due to rogue RA is the \textbf{VPN bypass}. Whenever you connect to a VPN, you will be part of a new network. Usually, this network is an IPv4 network and you're supposed to use that network for all the communication that should go in the VPN. If you're able to inject a rogue RA, you can fool the host and make it to use your own network, the one injected in the RA, instead of the VPN network. If the attacker is able to inject a new DNS or a new default gateway, he can sniff all client traffic, attempt MITM attacks, impersonate server/systems and capture presented user credentials, gain access into the other networks of the systems. The countermeasures are at network level. It means that your switches/routers should protect you against these types of attacks. The mechanism is not called anymore port security but is called \textbf{RA guard}. We can write some policies that states from which port the RAs have to come, i.e. it's the port where the default gateway is supposed to be. If we receive a RA from another port, we just discard it.
\paragraph{Router Advertisement flooding} We can also have a DoS attack, using the RA flooding. A host can potentially generate an IPv6 address for each prefix that he has received from RAs. The same reasoning is applied for the routing information.
\paragraph{DHCP rogue server} It's a situation in which an attacker is able to request all the available DHCP addresses. Once the addresses are gone, an attacker could use a rogue DHCP server to provide addresses to clients. Since DHCP responses include DNS servers and default gateway entries, the attacker can pretend to be anyone.
\section{Internet firewall}
The information that constitutes this section have been taken from Elizabeth D. Zwicky, Simon Cooper and D. Brent Chapman's book \cite{firewalls}. Firewall are a very effective type of network security. In building construction, a firewall is designed to keep a fire from spreading from one part of the building to another. In theory, an Internet firewall serves a similar purpose : it prevents the dangers of the Internet from spreading to your internal network. It serves multiple purposes : 
\begin{itemize}
\item it restricts people to entering at a carefully controlled point.
\item it prevents attackers from getting too close to your other defenses.
\item it restricts people to leaving at a carefully controlled point.
\end{itemize}
A firewall is a device that besides acting as a router, also contains and implements rules to determine whether packets are allowed to travel from one network to another. We have several ways to enforce the protection mechanism of the network :
\begin{itemize}
\item regulate which traffic is allowed
\item protect the traffic with encryption
\item monitor the traffic for "bad behavior"
\item monitor the hosts for "bad behavior".
\end{itemize}
The configuration will depend on the security policy to be fulfilled (see CIA targets). When we talk about firewall, we have to take some assumptions : 
\begin{itemize}
\item we have to know our policy stating what is allowed and not allowed.
\item we have to understand the good and bad traffic by its IP address, TCP port numbers, etc.
\item the firewall itself is immune to penetration.
\end{itemize}
\subsection{Security strategies}
It's important to understand some of the basic strategies employed in building firewalls and in enforcing security at your network.
\paragraph{Least Privilege} Basically, the principle of least privilege means that any object should have only the privileges the object needs to perform its assigned tasks and no more. Least privilege is an important principle for limiting your exposure to attacks and for limiting the damage caused by particular attacks.
\paragraph{Defense in depth} It doesn't depend on just one security mechanism, however strong it may seem to be; instead, install multiple mechanisms that back each other up. You don't want the failure of any single security mechanism to totally compromise your security.
\paragraph{Choke point} A choke point forces attackers to use a narrow channel, which you can monitor and control. In network security, the firewall between your network and the Internet is such a choke point; anyone who's going to attack your network from the Internet is going to have to come through that channel, which should be defended against such attacks. A choke point is useless if there is an effective way for an attacker to go around it.
\paragraph{Weakest links} Smart attackers are going to seek out the weak point of the security mechanism and concentrate their attentions there. We need to be aware of the weak points of our defense so that we can take steps to eliminate them, and so that we can carefully monitor those we can't eliminate. We should try to pay attention equally to all aspects of our security, so that there is no large difference in how insecure one thing is as compared to another. 
\paragraph{Fail-safe stance} If some network node fails, then it's should fail in such a way that they deny access to an attacker, rather than letting the attacker in. The failure may also result in denying access to legitimate users as well, until repairs are made, but this is usually an acceptable trade-off. The biggest application of this principle in network security is in choosing your network's stance with respect to security. In general, there are two fundamental stances that we can take with respect to security decisions and policies : the \textbf{default deny stance} to specify what we allow and prohibit everything else, and the \textbf{default permit stance} to specify what we prohibit and allow everything else.
\paragraph{Universal participation} If someone can simply opt out of your security mechanisms, then an attacker may be able to attack you by first attacking that exempt person's system and then attacking your network from the inside. In other words, the idea is to be able to convince all the users to choose the security mechanism used by a particular system.
\paragraph{Diversity of defense} It's closely related to depth of defense but takes matters a bit further; it's the idea that we need not only multiple layers of defense, but different kinds of defense. If we're not careful, we can create diversity of weakness instead of diversity of defense. If we have two different packet filters, one of them in front of the other, then using different products will help protect us from weaknesses in either one. If we have two different packet filter, each separately allowing traffic to come in, then using different products will merely make us vulnerable to two different sets of problems instead of one.
\paragraph{Simplicity} Simplicity is a security strategy for two reasons. First, keeping things simple makes them easier to understand; if you don't understand something, you can't really know whether or not it's secure. Second, complexity provides nooks and crannies for all sorts of things to hide in; Complex programs have more bugs, any of which may be a security problem.
\subsection{Packet filters}
The very first packet filter we want to describe is the \textbf{host based} one. It specifies the packets that can be received and sent (examples are the windows firewall and Linux iptables). The type of router used in a packet filtering firewall is known as a \textbf{screening router}. A screening router looks at packets more closely respect to how an ordinary router does. In addition to determining whether or not a packet can be forwarded to its destination, a screening router also determines whether or not it should. "Should" or "should not" are determined by the network security policy, which the screening router has been configured to enforce. On the other hand, it's very flexible; in fact we can permit or deny protocols by port number, but it's hard to allow some operations while denying others in the same protocol, or to be sure that what's coming in on a given port is actually the protocol we wanted to allow. In addition, it gives us no depth of defense. If the router is compromised, we have no further security. A screening router is an appropriate firewall for a situation where :
\begin{itemize}
\item the network being protected already has a high level of host security.
\item the number of protocols being used is limited, and the protocols themselves are straightforward.
\item we require maximum performance and redundancy.
\end{itemize}
Screening routers are most useful for internal firewalls and for networks that are dedicated to providing services to the Internet.
The most common way to have protection for our network is to use \textbf{dual-homed host} architecture. It's built around the dual-homed host computer, a computer that has at least two network interfaces. Such a host could act as a router between the networks these interfaces are attached to; it's capable of routing IP packets from one network to another. However, to use a dual-homed host as a firewall, we disable this routing function. Systems inside the firewall can communicate with the dual-homed host, and also external system can communicate with it, but they can't directly communicate with each other. Dual-homed hosts can provide a very high level of control. If we aren't allowing packets to go between external and internal networks at all, then we can be sure that any packets on the internal network that has an external source is evidence of some kind of security problem. In contrary, they are not high-performance devices. In fact, a dual-homed host has more work to do for each connection than a packet filter does, and thus needs more resources. An attacker who can compromise the dual-homed host has full access to our network, so we must guarantee an impeccable host security. A dual-homed host is an appropriate firewall for a situation where :
\begin{itemize}
\item traffic to the Internet is small.
\item traffic to the Internet is not business-critical.
\item no services are being provided to Internet-based users.
\item the network being protected doesn't contain extremely vulnerable data.
\end{itemize}
Another well known architecture is the \textbf{screened host}. Whereas a dual-homed host architecture provides services from a host that is attached to multiple networks, a screened host architecture provides services from a host that is attached to only the internal network, using a separate router. In this architecture the primary security is provided by packet filtering. This host is called \textbf{bastion host}, which is a hardened computer used to deal with all traffic coming to a protected network from outside. What does it means hardening ? Hardening is the task of reducing or removing vulnerabilities in a computer system :
\begin{itemize}
\item shutting down unused or dangerous services. Strengthening access controls on vital files.
\item removing unnecessary account permissions.
\item using "stricter" configurations for vulnerable components, such as DNS, FTP, etc.
\end{itemize}
There are some disadvantages to the screened host architecture. The major one is that if an attacker manages to break in to the bastion host, nothing is left in the way of network security between the bastion host and the rest of the internal hosts. A screened host architecture is appropriate when :
\begin{itemize}
\item few connections are coming from the Internet.
\item the network being protected has a relatively high level of host security.
\end{itemize}
Another architecture that we want to describe is the \textbf{screened subnet}. It adds an extra layer of security to the screened host architecture by adding a \textbf{perimeter network} that further isolates the internal network from the Internet. Why do this ? By their nature, bastion hosts are the most vulnerable machines on our network. Despite our best efforts to protect them, they are the machines most likely to be attacked because they are the machines that can be attacked. If, as in a screened host architecture, our internal network is wide open to attack from our bastion host, then this latter is a very tempting target. No other defenses are between it and our internal hosts. If someone successfully breaks into the bastion host in a screened host architecture, that intruder has hit the jackpot. So, by isolating the bastion host on a perimeter network, we can reduce the impact of a break-in on the bastion host. It gives an intruder some access but not all. The simplest type of screened subnet architecture, has two screening routers, each connected to the perimeter network. One sits between the perimeter network and the internal network (also known as \textbf{interior router}), and the other sits between the perimeter network and the external network (also known as \textbf{exterior router}). An attacker to break into the internal network with this type of architecture, must have to bypass both routers. Even if the attacker somehow broke into the bastion host, he would still have to get passed the interior router. There is no single vulnerable point that will compromise the internal network.\\A \textbf{demilitarized zone (DMZ)} is a special small network that is supposed to be protected from the outside, but it supposed to expose some services that a company intends to provide outside. DMZ is very often represented by means of firewall using the bastion host and the screening router. The first hop of the packets must be the bastion host and then it decides if they can be forwarded or not. We can segment the network in many different cases, for examples using a firewall to segment the network in several DMZs (also called VLAN).
\subsection{Stateless firewall}
In stateless firewall the packets are considered one by one. We receive a packet from one end of the filter, we look into it and decide if it can be forwarded or not. We can also take different decisions according to where the packet coming from (input or output interface). This is useful because we can protect the network from spoofed ip addresses. The packet filter operates at Data link, Network and Transport layer, observing source and destination addresses, and port numbers. So, image that we have :
\begin{enumerate}
\item our policy
\item we have expressed this policy in a formal language
\item and rewrote our policy in terms of the firewall syntax.
\end{enumerate}
The general mechanism is constituted by a set of rules to be checked from top to bottom. The first matching rule is applied and one implicit rule (also called the default rule) is assumed if no rule matched. Usually, the safest default decision is to block everything. In particular scenarios, is not enough to consider only a set of rules, because we can't control the direction of the traffic. For this reason, the packet filter system to consider the direction of the packets usually uses the TCP flags (for example the ACK flag).
\subsection{Stateful firewall}
A stateful firewall with respect to its stateless counterpart, offers state tracking of established connections. It is also called dynamic packet filtering because the behavior of the system changes depending on the traffic it sees. State tracking provides the ability to do things that we can't do otherwise, but it also adds complications. First, the router has to keep track of the state, increasing the load of traffic they should manage. This opens them to a number of DoS attacks, and means that if the router reboots, packets may be denied when they should have been accepted. Second, the router has to keep track of state without any guarantee that there's ever going to be a response packet (i.e. not all UDP packets have responses). For instance, DNS replies are supposed to arrive within 5 seconds, but reply times for name service queries across the Internet can be as high as 15 seconds; implementing the protocol specification (i.e. set of guidelines to avoid routers overload) will almost always deny a response that you wanted to accept.
\subsection{RAW, MANGLE, NAT and FILTER tables}
So far we have focused our attention on the FILTER table, but it's useful to notice that there are other tables in the iptables world such as :
\begin{itemize}
\item \textbf{RAW} : it's used for altering connection tracking, but by default it's not loaded since it's used for specific reasons.
\item \textbf{MANGLE} : it means to manipulate the packets, for example to change the TCP header, ToS, TTL, etc. It shouldn't be used for FILTERING or NAT.
\item \textbf{NAT} : it's used for network address translation, but we will see it in details in the next section. In particular, it handles the following special targets : DNAT, SNAT, MASQUERADE and REDIRECT. The DNAT is used in the PREROUTING chain to transform the destination IP of incoming packets, SNAT is used in the POSTROUTING chain to transform the source IP of outgoing packets, MASQUERADE is similar to SNAT, but it will change the outgoing packets inserting the IP address of the interface that in that particular moment the firewall has.
\item \textbf{FILTER} : in the previous section we have talked about stateless firewall, specifying that in some situation additional information are required in order to understand the traffic direction (egress filtering). Again, this is not enough, because somebody outside from our network can spoof some internal IP addresses. For this reason, we need to define rules based on from where packets are arriving (ingress filtering). Ingress filtering is a technique used to ensure that incoming packets are actually from the networks from which they claim to originate. This can be used as a countermeasure against various spoofing or DoS attacks. Instead, egress filtering is the technique opposite to the ingress filtering one. In fact, it monitors and potentially restricts the flow of information outbound from one network to another. They ensure that malicious traffic never leaves the internal network. Furthermore, we have some problems here :  we take into account only a small number of parameters, we can't inspect the payload of TCP packets and there are some limitations about logging ability, no authentication facilities, and so on.
\end{itemize}
There exist a priority between the previous tables : in particular, if we represent with $A > B$ the meaning that $A$ has higher priority with respect to $B$, then we have RAW $>$ MANGLE $>$ NAT $>$ FILTER. We can also specify additional chains in the different tables, in order to perform additional checks on the packets. It's very similar to the standard chain mechanism, but we can jump to a different chain within the same table.
\subsection{Other types of firewalls}
In this section we want to briefly describe some other types of firewall we could facing along the way.
\paragraph{Application-level proxy} An application-level proxy is one that knows about the particular application it is providing proxy services for; it understands and interprets the commands in the application protocol. 
\paragraph{Circuit-level proxy} A circuit-level proxy is one that creates a circuit between the client and the server without interpreting the application protocol. The advantage of a circuit-level proxy is that it provides service for a wide variety of different protocols. Most circuit-level proxy servers are also generic proxy servers; they can be adapted to serves almost any protocol. In contrary, not any protocol can be easily handled by a application-level proxy, because some of them could require some application-level knowledge (like FTP protocol). The disadvantage of a circuit-level proxy server is that it provides very little control over what happens through the proxy. Like a packet filter, it controls connections on the basis of their source and destination and can't easily determine whether the commands going through it are safe or even in the expected protocol. In general, circuit-level proxies are functionally equivalent to packet filters.\\\\The most extreme version of the first approach is an application like Sendmail, which implements a store-and-forward protocol. The most extreme version of the second approach is an application like plug-gw, which accepts all data that it receives and forwards it to another destination. When we use the application gateway, for every incoming packet we verify the client application and then verify the user. While in the circuit level, for every incoming packet we consider the user authentication.
\section{Network Address Translation}
The internal network uses private IP addresses provided by IETF, and can change them for hosts/devices within this network without notifying the world outside. While IP addresses for hosts in the external network are unique and valid in this environment as well as in private networks, the addresses for hosts in the private network are unique only within this private network and may not be valid in the external network. The IP address binding in some cases may extend to transport level identifiers such as TCP/UDP ports. The address binding is done at the start of a session, and a traditional NAT device would allow hosts within a private network to transparently access hosts in the external/public network, in most cases. There are two variations to traditional NAT : 
\begin{itemize}
\item \textbf{Basic NAT} : a block of public IP addresses is set aside for translating the addresses of hosts within a private domain as they originate sessions to the external domain. In fact, translation occur for both outbound and inbound packets. For outbound packets from the private network, the source IP address and related fields such as IP, TCP, UDP and ICMP header checksums are translated. For inbound packets, the destination IP address and the checksums are translated.
\item \textbf{Network Address Port Translation (NAPT)} : it also translates transport identifiers, e.g. TCP and UDP port numbers as well as ICMP query identifiers. This permits the transport identifiers of a number of private hosts to be multiplexed into the transport identifier of a single public IP address. The NAPT allows a set of hosts to share a single external address. For most of the SOHO routers, the private network usually relies on a single IP address, supplied by the ISP to connect to the Internet, and can change ISPs without changing the private IP addresses of the devices within the network, since these devices inside the network are not explicitly addressable by the external network. This latter point is also a security advantage. 
\end{itemize}
The NAPT maps the private source address and source port number to a public source address and a public-side port number at the NAPT router for outgoing packets. Incoming packets, addressed to this public address and port pair, are translated to the corresponding local address and port. The NAPT translation table provides a one-to-one mapping entry between the two pairs : 
\textit{private source IP address:port number} and the \textit{public IP address:new port number}. With outgoing datagrams, the first pair is replaced with the second one, and remote client will reply using this latter pair as the destination transport address. For incoming packets, the previous process is reversed. One of the more pressing problems encountered is that in which NATs commonly enforce an application model where a local, private, hidden host must initiate a transaction in order to create a hole in the NAT in order to allow the packets of the external host back into the private network. Some applications may wish to undertake a "referral", in which the correspondent host on the external side may want to pass the externally presented address and port details of the local host to a third party in order to start a further part of the transaction. By default, NAPT routers block all incoming requests and only allow the response packets of outgoing requests to pass through it as a result of the available mapping entries. Many applications have had problems with NAPT in the past in their handling of incoming requests. There are four major methods used for handling the connection to a server that exists between a private network and the outside network :
\begin{itemize}
\item Application Level Gateways (ALGs)
\item Static port forwarding
\item Universal Plug and Play (UPnP), Internet Gateway Device (IGD) protocol
\item Traversal Using Relays around NAT (TURN).
\end{itemize}
In an IP packet we can translate source or destination address using two different methods :
\begin{itemize}
\item \textbf{Source NAT (SNAT)} : it means we want to change the IP source address of the host that generates the packet with a public one. We can easily notice that source translation happens when a packet wants to go out from a private network. This is also called masquerade, because it acts like a mask applied to packet source address, that will be removed when the answer arrives. When the response packet arrives to the NAT router, we also apply a DNAT in order to translate the public destination address to a private destination address. In this section we are referring to SNAT, since first is applied a masquerade to the source address and then to the destination address. Typically, the considered router has a table called \textbf{NAT table} that uses to store several mappings between private and public addresses. We want to underline that the data stored in the NAT tables are temporary, i.e. when the connection will be closed, NAT table entries can be removed. 
\item \textbf{Destination NAT (DNAT)} : we need the destination NAT when we want to have an internal host to receive a connection from the outside. We have a source client that wants to reach a server within our internal network. In other words, the server address isn't a private IP address but will be a public IP address that is the one assigned to the firewall external interface. When the firewall receives a packet from the outside that wants to reach our server, it has to translate the packet destination address into the server private IP address. This is also called \textbf{Port Forwarding} or \textbf{Virtual Server}. The idea is the following : whenever the firewall receives a connection request on the external interface, according to the requested port it will forward to a different host.
\end{itemize}
\section{Virtual Private Networks}
A VPN is a virtual network built on top of an existing network infrastructure, which can provide a secure communications mechanism for data and other information transferred between two endpoints. This new network typically is based on the use of encryption deciding how and where to perform it and what parts of the communications should be encrypted. The VPN main goal is the usability, so that has to be in some way easy to use and to be exploited. This goal can be measured with the following metrics :
\begin{itemize}
\item \textbf{Transparency} : it means how much burden the users of the VPN should sustain to use it.
\item \textbf{Flexibility} : how easily can be used the VPN and how many uses we can have of the VPN.
\item \textbf{Simplicity} : how easy can be to deploy and use the VPN.
\end{itemize}
The VPN security goals are : 
\begin{itemize}
\item \textbf{Confidentiality of data} : it means that we don't want that other guys access the data.
\item \textbf{Integrity of data} : it means that any of the two endpoints can recognize that the transmission was done without some errors or changes.
\item \textbf{Peer authentication} : we can know the authenticity of the data, i.e. the data that we receives is coming from the original sender.
\end{itemize}
We can also see some extended features about a VPN such as :
\begin{itemize}
\item \textbf{Replay protection} : we can distinguish between fresh and old data.
\item \textbf{Access control} : we can limit the access to the network resources to the users that leverages the network.
\item \textbf{Traffic analysis protection} : we can use some techniques in order to avoid some kind of analysis of encrypted data that can really reveal some details.
\end{itemize}
The \textbf{site-to-site} security means that we are using as an insecure channel Internet, to provide a virtual circuit between two endpoints that are two routers that trying to bind together two sites of the same institution. The idea is that all the traffic that is supposed to be routed between these two sites, instead of going as they are into the Internet, they are transformed and injected in this channel that they have established in Internet. We can also have a different type of security called \textbf{host-to-site} security, where one the two endpoints is a single host. The idea is that we again we are using Internet to provide the access to join a host to the network of one site, and the virtual circuit between the host and gateway provides and guarantee the security properties. Another type of security is the \textbf{host-to-host} security, where who is performing the encryption are the two endpoints in the network. We can also distinguish where we want to apply the security. In particular, referring to TCP/IP model we can implement the security at :
\begin{itemize}
\item \textbf{Physical layer} : it means that we are securing a wire. From the point of view of the flexibility is very bad, because is only a point-to-point security and thus hard to adapt to other contexts. From the point of view of the transparency, we are not changing anything from the point of view of the applications, because the encryption is only at physical level. A physical VPN is the less common to be used.
\item \textbf{Data link layer} : we need to remember that the data link layer in some way supposed to carrying in the payload all the data of the application level and all the headers of the previous layers. In this case the confidentiality is on the link, i.e. all the data exchanged on this virtual cable are secure. Also in this case we have full transparency, so the users won't notice in any case that there is the VPN, but isn't flexible because it's again a n-to-point security.
\item \textbf{Network layer} : it means includes some IP packets (not really IP packets because they're encrypted) in the packets header. It's very useful, because we can keep the already deployed network and hide the contents of IP packets (this type of VPN is implemented with IPsec). It's very transparent, flexible and simple for site-to-site security. It's not so simple for host-to-site security, because from the point of view of the transparency we need some extra setup in the host.
\item \textbf{Transport layer} : in this case the protection is end-to-end between processes. It's very good for site-to-site and host-to-site security. It's very good also for access control and transparency (this type of VPN is implemented with SSL/TLS).
\item \textbf{Application layer} : it's not really a VPN, but we are making two applications to communicate in a secure way. This is the standard HTTPS realization. It's very good from the point of view of the transparency, protection of the data but it isn't flexible because we have to use a different VPN for each application we want to use. This is a common implementation used for host-to-host security.
\end{itemize}
\subsection{SSL Tunneling}
We've introduced the concept of tunneling, which means that we have established a tunnel between two endpoints and it provides some guarantees to all the data that pass through it such as encryption and authenticity. We can realize two types of security, as reported by NIST's special publication \cite{ssl_vpn} :
\begin{itemize}
\item \textbf{transport site-to-site} : we can image that the sender PDU is encapsulated in another PDU (IP packet) in order to allow it to cross the network that connects the two sites. In particular, the encapsulation takes place in the edge router on the source site, while the decapsulation in the edge router on the destination site.
\item \textbf{tunneled site-to-site} : we can also consider site-to-site security with encryption, i.e. before encapsulating the PDU to be transported in another PDU, it's encrypted so that its contents cannot be seen or changed by hosts on the route. The encryption can take place in the edge router on the source site, whereas the decryption in the edge router on the destination site.
\end{itemize}
The tunnel is the way in how we realize a VPN. Lets start with an introduction with \textbf{Secure Socket Layer (SSL)}. SSL was originally designed by Netscape to offer security for client-server session. There are many version of SSL, and starting from the version 3.0 it became TLS standard with small changes. Clearly, the security is the Transport layer. Since it works at transport layer, we can have at the lower layer of the stack exactly the same protocols that we have in the standard Internet. Lets review shortly how HTTPS works (i.e. HTTP on top of TLS). When you connect to a website that provide HTTPS security, remember that the only thing that it does it's uses HTTP on top of TLS. This means that before using HTTP, instead of using the standard TCP, we use TLS to establish the secure channel. The first packet is the one intended to obtain the first part of information that represent the certificate of the server. It's important to have the server certificate because we want to be really sure that the key we will use to exchange the data is the one that belongs to the website. The certificate is a way to bind the key of the website to the domain name of the website. It has a public key, which is the one we have to use to encrypt the communication that has to go to the server. This binding is guaranteed by a digital signature of our Certificate Authority. We can use the server public key to verify the digital signature, and if this check is positive we are sure that this is the real public key of the server. Now, we can generate a symmetric key that is supposed to be used during the secure communication. We can encrypt this key with the asymmetric key of the website, and sent to insecure Internet to reach the website. Then the website can decrypt it using its own private key. Now, both parties are agrees on a secure key and it can be used to cipher a channel between them. Between HTTP and TCP we insert SSL protocol, also called \textbf{Record Protocol}. SSL/TLS Record protocol secures application data using the keys created during the handshake. It's  responsible for securing application data and verifying its integrity and origin. It manages the following operations :
\begin{itemize}
\item divides outgoing messages into manageable blocks ($\leq 2^{14}$ bytes) and reassembles incoming messages.
\item compress outgoing blocks and decompress the incoming ones.
\item applies a MAC to outgoing messages and verifying the incoming ones using it.
\item encrypt outgoing messages and decrypt the incoming ones.
\end{itemize}
When the RP is completed, the outgoing encrypted data are passed down to the TCP layer for transport. Before being able to use HTTP we need some extra tools such as :
\begin{itemize}
\item \textbf{SSL handshake} : used to authenticate the server and to agree on encryption keys and algorithms.
\item \textbf{SSL change cipher spec} : selects agreed key and encryption algorithm until further notice.
\item \textbf{SSL alert} : transfers information about failures.
\end{itemize}
The SSL handshake is constituted by four main phases to establish the parameters of a secure connection :
\begin{enumerate}
\item \textbf{Hello} : the first step is the establishment of security capabilities. The client sends a list of possibilities, in order of preference. Server selects one, and inform the client of its choice. Parties also exchange random noise for use in key generation.
\item \textbf{Server authentication and key exchange} : server executes selected key exchange protocol (if needed). Server sends authentication information (e.g. X.509 certificate) to client.
\item \textbf{Client authentication and key exchange} : client executes selected key exchange protocol (mandatory). Client sends authentication information to Server (optional).
\item \textbf{Finish} : shared secret key is derived from pre-secrets exchange in 2) and 3) Change cipher protocol is activated. Summaries of handshake protocol are exchanged and checked by both parties.
\end{enumerate}
A TLS packet is obtained through a fragmentation of the PDUs, in turn compressed, with the addition of a keyed MAC, then encrypted and finally the addition of an header that indicates the application protocol in use. The SSL/TLS security capabilities are conventionally expressed by a descriptive string that has this format :
\begin{center}
Version\_KE-algorithm\_WITH\_encryption-algorithm\_encryption-mode\_crypto-checksum-algorithm,
\end{center}
where :
\begin{itemize}
\item Version : it's the SSL/TLS version
\item IKE\_algorithm : it's the Key Exchange algorithm
\item encryption-algorithm : it's the chosen encryption algorithm.
\item encryption-mode : it's the mode of block encryption (only for block cipher).
\item crypto-checksum-algorithm : it's the chosen cryptographic checksum algorithm.
\end{itemize}
For example, we can have TLS\_RSA\_WITH\_AES\_128\_CBC\_SHA. A VPN realized with SSL brings to two models :
\begin{itemize}
\item \textbf{SSL Portal VPN} : it allows a user to use a single standard SSL connection to a website to securely access multiple network services. The site accessed is typically called a portal because it has a single page that leads to many other resources. The remote user accesses the SSL VPN gateway using any modern web browser, identifies himself to the gateway using an authentication method supported by the gateway, and then a web page is presented that acts as the portal to the other services. Since they works also with browsers that doesn't allows active content, they are accessible to more users than SSL tunnel VPNs.
\item \textbf{SSL Tunnel VPN} : it allows a user to use a typical web browser to securely access multiple network services through a tunnel that is running under SSL. SSL tunnel VPNs requires that the web browser be able to handle specific types of active content and that the user be able to run them. This tunnel gives to the user full access to services on the network protected by the VPN gateway. Due to the active content requirement, SSL tunnel VPNs may be accessible to fewer users than SSL portal VPNs.
\end{itemize} 
Most SSL VPNs offers one or more core functionalities such as : \textbf{Proxying} with which an intermediate device appears as true server to the clients, \textbf{Application Translation} to convert information from one protocol to another, \textbf{Network Extension} to provide a partial or complete network access to remote users, typically via Tunnel VPN. The types of services that grants SSL are the following :
\begin{itemize}
\item \textbf{Authentication} : because we are using certificates.
\item \textbf{Encryption} : via the use of the SSL/TLS protocol.
\item \textbf{Access control} : because we can also enforce some types of access regulation when we have the client authentication.
\item \textbf{Endpoint security controls} : we can also enforce some controls on the endpoints, for example to be sure that they doesn't have an old version of the OS.
\item \textbf{Intrusion prevention} : it's used to evaluate decrypted data for malicious attacks. 
\end{itemize}
\subsubsection{SSL/TLS Heartbeat}
Whenever you establish the TLS, there is a channel established. Now, if the two endpoints don't use the channel for an extended time then you have to close the session. The idea is that we have an extension that allows to keep this session alive, simply using a timer that expires when there is no traffic exchange in the channel. Whenever there is this timer, we exchange a small \textbf{Heartbeat packet}, which avoid the re-negotiation of the security parameters for that session. In particular, we introduce two types of packets : \textbf{Heartbeat Request} and \textbf{Heartbeat Response}. Whenever an endpoint receives an Heartbeat Response simply reset the timer. As a protection against a replay attack, Heartbeat Request packets include a payload that must be returned without changes by the receiver in its Heartbeat Response packet.  The problem with this mechanism is the following : in the implementation of the answer of the response, happen that the message was placed in memory, it simply read the length of the payload and it get from the memory that amount of bytes where this packet was stored. If an attacker manages to send an Heartbeat Request with a payload size much greater than the original one, then he's able to get some bytes from the receiver's memory, that may contains sensible information. This bug is also called \textbf{Heartbleed bug}.
\subsection{Device placement}
Another important point is the place supposed to be where the VPN device is located, since it affects the VPN features such as security, functionality and performance. We can place a VPN device in the following places :
\begin{itemize}
\item \textbf{Firewall} : whenever the traffic reach the firewall become unencrypted and can reach the internal network or the DMZ. We have no holes in the firewall between external VPN device and internal network, the traffic between the device and internal network must go through the firewall and it's very simple to configure. The main disadvantages are the following : limited to VPN functionality offered by firewall vendor, adding new VPN functionality to the firewall can introduce vulnerabilities, external hosts can access the firewall via port 443 (since it must be open on the firewall external interface to allow clients to initiate a connection.
\item \textbf{Internal network} : the traffic goes in the firewall and reaches the internal VPN device. Also in this case we haven't holes in the firewall between the VPN device and internal network, the VPN traffic is protected from attacks by machines in DMZ since it's encrypted until reaches the VPN device in the internal network, it's very easy to configure. The problems here are that the VPN traffic isn't analyzed by the firewall, the network is compromised if the VPN device is compromised.
\item \textbf{Single-interface VPN device in DMZ} : it's one of the most common way to realize a VPN. This is useful because we have the possibility to explore the traffic that is outside the VPN using intrusion detection system, but it's a bit more complex to be configured in the firewall, and we have unencrypted traffic in the DMZ.
\item \textbf{Dual-interface VPN device in DMZ} : in this case the VPN device within the DMZ can be directly connected to the internal network passing through the firewall. Now the unencrypted traffic to internal hosts is protected from other hosts in the DMZ, but it's more complex to be configured in the firewall because we have different sources of traffic from the DMZ. 
\end{itemize} 
\subsection{IPsec}
IPsec with respect to SSL VPN is placed on the network layer. It's a collection of protocols that assist in protecting communications over IP networks. IPsec protocols work together in various combinations in order to provide protection for communications. The three main IPsec protocols, as reported by NIST special publication \cite{ipsec_vpn}, are the following :
\begin{itemize}
\item \textbf{Authentication Header (AH)} : it provides integrity protection for packet headers and data, as well as user authentication, adding its own header. It can optionally provide replay and access protection. AH cannot encrypt any portion of packets. AH can authenticate portions of packets that ESP cannot. AH has two modes :
\begin{itemize}
\item \textbf{Transport mode} : AH doesn't create a new IP header. It authenticate IP payload and selected parts of IP header and IPv6 extension headers.
\item \textbf{Tunnel mode} : AH create a new IP header for each packet. It authenticate the whole inner IP packet and selected parts of outer IP header and outer IPv6 extension headers. 
\end{itemize}
\item \textbf{Encapsulated Security Payload (ESP)} : it provides encryption for packet payload data and optionally authentication, adding its own header and trailer. ESP also has two modes : 
\begin{itemize}
\item \textbf{Transport mode} : ESP uses the original IP header, instead of creating a new one. It encrypt IP payload + any IPv6 extension headers after ESP header.
\item \textbf{Tunnel mode} : ESP creates a new IP header for each packet. It encrypt inner IP packet.
\end{itemize}
\item \textbf{Internet Key Exchange (IKE)} : its purpose is to negotiate, create and manage \textbf{security associations (SAs)}. SA is a generic term for a set of values that define the IPsec features and protections applied to a connection. Each SA is identified by a Security Parameters Index (SPI), which is a 32-bit integer chosen by sender, that enables receiving system to select the required SA, a destination unicast address and a Security Protocol Identifier (AH or ESP).
\end{itemize}
Often AH and ESP are used together to provide both authentication and encryption. There are several combinations of AH + ESP, for example we apply first AH in transport mode and then ESP in tunnel mode, or both in tunnel mode and so on so forth. TLS is much more flexible respect to IPsec, because is in the upper levels. TLS also provides application end-to-end security (best for web applications, for example based on HTTPS). IPsec has to run in kernel space and thus is more complex to manage.
\section{Proxy}
A very first introduction is due to Ari Luotonen and Kevin Altis's article \cite{www_proxies}. The primary use of proxies is to allow access to the Web from within a firewall. A proxy is a special HTTP server that typically runs on a firewall machine. The proxy waits for a request from inside the firewall, forwards the request to the remote server outside the firewall, reads the response and then forwards back it to the client. In the usual case, the same proxy is used by all the clients within a given subnet. This allow the proxy to perform efficient caching of documents that are requested by a number of clients. This cuts down on network traffic costs since many of the documents are retrieved from a local cache once the initial request has been made. An \textbf{application-level proxy} makes a firewall safely permeable for users in an organization, without creating a potential security hole through which an attacker can get into the organization network. For Web clients, the modifications needed to support application-level proxying are minor, i.e. there is no need to compile special versions of Web clients with firewall libraries, the standard out-of-the-box client can be configured to be a proxy client. For example, clients without DNS can still use the Web. The proxy IP address is the only information they need. Proxying allows for high level logging of client transactions, including client IP address, date and time, URL byte count and success code. It is also possible to do filtering of client transactions at the application protocol level. The proxy can control access to services for individual methods, host and domain, etc. Application-level proxy facilitates caching at the proxy. In fact, caching is more effective on the proxy server than on each client.This saves disk space since only a single copy is cached, and also allows for more efficient caching of documents that are often referenced by multiple clients as the cache manager can predict which documents are worth caching for a long time and which are not. A caching server would be able to use "look ahead" and other predictive algorithms more effectively because it has many clients and therefore a larger sample size to base its statistics on. When a normal HTTP request is made by a client, the HTTP server gets only the path and keyword portion of the requested url. The requested path specifies the document or a CGI script on the local filesystem of the server, or some other resource available from that server. When a client sends a request to a proxy server the situation is slightly different. The client always uses HTTP for transactions with the proxy server, even when accessing a resource served by a remote server using another protocol like FTP. However, instead of specifying only the pathname and possibly search keywords to the proxy server, the full url is specified. In this way, the proxy server has all the information necessary to make the actual request to the remote server specified in the request url, using the protocol specified in the url. From this point of view, the proxy server acts like a client to retrieve the required document. However, the proxy will create an HTTP reply containing the requested document and will send it to the client. The thing we have to keep in mind here is that a complete proxy server should speak all the Web protocols, in order to serve the clients requests. In fact, proxy that only handle a single Internet protocol such as HTTP are also a possibility, but a Web client would then require access to other proxy servers to handle the remaining protocols. Other benefits of forward proxy are : 
\begin{itemize}
\item \textbf{Authentication} : it means that only authenticated users can leverages the proxy to reach Internet.
\item \textbf{Authorization} : it means that only the authorized users by the proxy can use it to reach Internet.
\item \textbf{Whitelisting and Blacklisting} : the proxy can write a whitelist, which include the websites that are allowed to be reached. In contrary, it can also write a blacklist to deny the access to particular websites.
\item \textbf{Caching} : it means that we can store the documents into the proxy file system. For further use, it wouldn't be necessary to connect to the remote server again to request the same document. But we need to keep in mind some questions such as how long is it possible to keep a document in the cache and still be sure that it is up-to-date ? or how to decide which documents are worth caching and for how long ? There are several solutions like using \textbf{HTTP HEAD request}, i.e. we provide a request that is very similar to HTTP GET request but the answer that we obtain is equal to the one that we would obtain using the GET request, but only in the header. If the resources that the proxy server has in cache are the same that the origin server would have provided, then we don't need to download again. The other option is to use \textbf{If-Modified-Since} header field, i.e. include a timestamp in the header, so that the server compares the value of this field with the last modification of the original resource, and provides us the content only if the timestamps are different.
\end{itemize} 
If we instead of using HTTP we want to use HTTPS, how can we use a forward proxy ? We can't use HTTP because HTTPS includes secure keys that cannot be in plaintext. To solve this problem we can use a clever solution called \textbf{HTTP tunneling}, which is realized with a method called \textbf{HTTP CONNECT}. It's again a request that the client does to the proxy, but now instead of requesting a resource to the origin server, it requests a connection. It means that the proxy will establish a TCP connection with the origin server and will forward all the bytes that client and server are supposed to exchange. Clearly the HTTP CONNECT will be allowed only after authentication and also if the website is in the whitelist, and so on. In this case, the proxy will be unable to intercept the traffic, because the two endpoints who is performing encryption are the client and the origin server. 
\subsection{Other types of proxy}
In this section we want to describe several types of proxy that we can implement.
\paragraph{Content-filtering proxy} It's used to perform sanity check on the data requested by users. A content-filtering web proxy server provides administrative control over the content that may be relayed in one or both directions through the proxy. It will often support user authentication to control web access. It also produces logs, either to give detailed information about the urls accessed by specific users or to monitor bandwidth usage statistics. 
\paragraph{Anonymizer proxy} Another class of forward proxy is the one called \textbf{anonymizer proxy}. Since we are hiding the original client IP, we can use this feature to really hide the IP header. The idea is that we are not really helping an internal host to go outside, but we are using a public IP address to hide its own IP. Since the proxy is an intermediary that will change the original IP address, it will act as a privacy shield. The client needs to be aware of the presence of this type of proxy. In this case, the origin server will see the requests coming from the proxy address rather than the user's IP address. It has a lot of advantages like hide the real client identity, but the traffic between itself and the client can be eavesdropped, because it doesn't encrypt the replies intended for the host that made the request.
\paragraph{SSL forward proxy} Exploiting the proxy mechanism we can enforce another mechanism to protect our network which is called \textbf{SSL forward proxy}. The idea is that we can decrypt and inspect the traffic (SSL/TLS or HTTPS traffic) before this traffic reaches the original client. The proxy decrypt the content before it is received by the client, so that the proxy can check if there is something strange in the communication. Since we are using encryption, we need a way to decrypt the traffic. The idea is that the proxy will be the real one that realizes the TLS connection. In particular, the proxy will establish two different TLS session, one with the client and one with the server. The proxy act as an intermediary between client and origin server, decrypting the traffic coming from the client and re-encrypting it to send to the origin server (same reasoning is applied for the responses). 
\paragraph{Reverse proxy} Another type of proxy is the \textbf{reverse proxy}, which applies the concept of being a proxy, but it's close to the server (it can be seen as the forward proxy counterpart, which is placed close to the clients). In forward proxy scenario, the proxy itself is seen as server from the point of view of the clients. Whereas in reverse proxy scenario, the proxy is seen as client from the point of view of the origin server. The fundamental idea of a reverse proxy is that it receives the requests of the original clients, instead of the origin server, and really forwards them to the actual destination (the origin server). A reverse proxy can have many functions such as load balancing, cache static content, compression, application level controls (e.g. Modsecurity), TLS acceleration, and so on. ModSecurity is an open-source web application firewall, originally designed as a module for the Apache HTTP server, has evolved to provide an array of  HTTP request and response filtering capabilities along with other security features across a number of different platforms. To detect threats, the ModSecurity engine is deployed embedded within webserver or as a proxy server in front of a web application. This allows the engine to scan incoming and outgoing HTTP communications to the endpoint. For example for implement application level controls we can use the so called \textbf{Web Application Firewall}, which inspect the HTTP traffic and prevents attack such as SQL injection, XSS and other types of security issues. It can block application input/output from detected intrusions or malformed communication or block contents that violate policies. Furthermore, it can detect whether an unwanted protocol is being provided through a non-standard port or whether a protocol is being abused in any harmful way. Another mechanism that we can enforce in a reverse proxy is called \textbf{TLS acceleration}. The SSL/TLS handshake process uses digital certificates based on asymmetric or public key encryption technology. The public key encryption is very secure, but it requires a lot of resources, negatively affecting the performance (a common problem is the SSL bottleneck). Possible solutions to the previous problem are : 
\begin{itemize}
\item \textbf{SSL acceleration} : using a proper hardware to perform the public key encryption.
\item \textbf{SSL offloading} : use a dedicated server that only perform the SSL handshake. It can be performed in two ways : 
\begin{itemize}
\item \textbf{SSL termination} : the proxy decrypts the SSL-encrypted data and then sends it to the server in an unencrypted state.
\item \textbf{SSL forwarding} : the proxy intercepts and decrypts SSL-encrypted traffic, examines the contents to ensure that it doesn't contain malicious code and then re-encrypts it before sending it to the server. This only allows inspection of coming encrypted traffic before it reaches the server, in order to prevent application layer attacks hidden inside.
\end{itemize}
\end{itemize}
\paragraph{SOCKS proxy} SOCKS is an Internet protocol that permits client-server applications to transparently use the services of a circuit-level network firewall. It provides strong user authentication and host name resolution. Client hosts must be made aware of the fact that they are using a circuit-level proxy since this proxy is required in order to install SOCKS client software. It performs a number of functions for client hosts residing behind a firewall. When hosts need to access exterior servers, hosts can connect to a SOCKS proxy server for authentication. The proxy server controls the client's access to an external server using an access control list (ACL), and relays the request on to the requested clients outside the firewall. This gateway inspect and realys TCP/UDP/ICMP packets, and filtering is based upon the state, i.e. a session. This gateway does not inspect the contents of payloads, and is weaker, but faster than an application-level gateway. Its real challenge is proxying UDP and ICMP packets, since each packet may be a separate transaction.
\paragraph{Transparent proxy} Another type of proxy is the \textbf{transparent proxy}. With the standard proxy we have to explicitly make the requests to the proxy. We go to the proxy and tell to it which server we want to access. We need to properly configure every software that we want to use to connect with the proxy. The idea of the transparent proxy is to avoid this kind of setup from the point of view of the client. The client needs to use the standard requests without using the proxy requests, but we want to really realizes these requests from a proxy. It's like our firewall performing the requests instead of us, something really transparent to the users. It will establish two SSL/TLS sessions, one with the client and one with the server. Suppose we have established a connection between client and server. When the proxy server sees a request for the original destination, instead of forwarding the packet to the intended destination, it receives the request itself and will perform the request instead of the original client. This mechanism is similar to the NAT one. We have some problems with the transparent proxy : with the standard HTTP request we can intercept the packets only if the IP address is the one of the server. If we don't known the host name we have to observe it in the HTTP requests. If we use HTTPS it's impossible for the transparent proxy to really read the name of the host server without breaking the encryption (same issue of reverse proxy and SSL bump). The client won't be a real transparent proxy, because if we use the TLS encryption with the standard proxy we will use the HTTP CONNECT, we perform the tunnel and actually the two endpoints of the communication will be the client and the server. With the transparent proxy, our client will have a security alert, thus we can't use this mechanism in a real transparent way. The client is supposed to communicate with the server and it expect to use the certificate of that server. Since we are using SSL bump, we are receiving a certificate that issued by our proxy, so there is a mismatch between the server certificate domain name and the expected server domain name, then we have an alert in the client browser. For this reason, typically the transparent proxy is applied using the \textbf{policy based routing}. It's different from the traditional routing. Traditional IPv4 routing is summarized as "All routing is a destination-driven process." When a router looks at an IPv4 packet it cares only about the destination address in the header of the packet. It uses this destination address to make a decision on where to forward the packet. Instead, the policy based routing provides routing capability based on other information such as Quality of Service (e.g. in a subscription scenario, if a user pay the complete fee then he can use the fastest links, otherwise he will only be able to use slower links), source routing, traffic shaping, etc. For example we can use the policy based routing for the transparent proxy saying that any tcp connection that has destination port 80 must be forwarded to the proxy, all the other can go directly to the intended destination. We need a different routing table that actually changes the next hop with the proxy. This can be done via IPsec using the packet marking mechanism. This is an extra step to realize this mechanism of transparent proxy to make an extra look to the packets we are exchanging. Once we have this setting, we can perform some inspections at the application layer using the \textbf{Internet Content Adaption Protocol (ICAP)}. It's a protocol aimed at providing simple object-based content vectoring for HTTP services. In essence, ICAP is a lightweight protocol for executing a remote procedure call on HTTP messages. It allows ICAP clients (e.g. our proxy) to pass HTTP messages to ICAP servers for some sort of transformation or other processing. The server executes its transformation service on messages and sends back responses to the client, usually with modified messages. This type of RPC is useful in several ways :
\begin{itemize}
\item simple transformation of content can be performed near the edge of the network instead of requiring an updated copy of an object from an origin server.
\item surrogates or origin servers can avoid performing expensive operations by shipping the work off to other servers instead. For example check if a file contains viruses.
\item firewalls or surrogates can act as ICAP clients and send outgoing requests to a service that checks to make sure the URI in the request is allowed. For example in a system that allows parental control of web content viewed by children.
\end{itemize}
Now, we want to talk about a very critical issue we can have when using a proxy, the \textbf{HTTP certificate dilemma}. We known that HTTPS cannot be read, so we need a mechanism to realize this removal of the encryption. If we want to read a client HTTPS traffic, we need to use a MITM attack, i.e. pretend to be the real server and be the termination of the SSL/TLS connection. This is also called \textbf{SSL bump}, which consists in using the requested host name to dynamically generate a server certificate and then impersonate the named server. This certificate will be signed by the proxy itself. Now we have a problem, because the certificate is related to the host name. Suppose we have a web server that host multiple websites. The real server will provide to the original client the certificate of virtual host that the client really request, but the HTTP request is sent after the TLS establishment. In other words, we have to receive a certificate with a name of the server before we tell to the server the name of the host we want to connect to. If we want to use HTTPS, the SSL/TLS connection needs a certificate, but we have not a way to send to the original server the name of the server that we want to connect. For this reason, we need to use the \textbf{Server Name Indication (SNI)}, an extension to TLS, by which a client indicates which hostname it's attempting to connect to at the start of the handshaking process. It's in plaintext, thus some malicious host  could eavesdrops it. This allow a server to present multiple certificates on the same IP address and TCP port number and hence allows multiple secure websites to be served by the same IP address without requiring all those sites to use the same certificate. In case that the server who are requiring a domain uses a wildcard certificate protection, a possible solution is to use the \textbf{domain fronting}, with which we specify a domain in the handshake phase (that will be read from everybody since it's in plaintext) and use another one when the TLS tunnel has been established. In this way nobody can intercept or read this latter domain, since it's encrypted.
\section{Intrusion Detection and Prevention System}
An intrusion detection/prevention system (IDS/IPS) is another element in the arsenal which is employed to provide deep packet inspection at the entrance of important network. The IDS/IPS provides deep packet inspection for the payload, IDS is based on out-of-band detection of intrusions and their reporting, and IPS is in-band filtering to block intrusions. IDS is performed through a wire tap and is clearly an out-of-band operation. In contrast, IPS is performed inline. We need to be aware about the possible alarms that can be raised and not. In fact an alarm can be classified in one of the following categories :
\begin{itemize}
\item \textbf{True positive} : the system raises an alarm since it occurs a real intrusion.
\item \textbf{False positive} : the system raises an alarm when there isn't any intrusion at all. In other words, the anomaly actually detected was benign.
\item \textbf{False negative} : the system doesn't raise any alarm, due a real intrusion was not detected. In other words, the anomaly not detected was malign.
\item \textbf{True negative} : the system doesn't raise any alarm, due it doesn't occurs any intrusion.
\end{itemize}
An IDS/IPS can also provide additional functionalities such as recording information related to observed events, notifying security administrators of important observed events, producing reports. The typical workflow of an IDS/IPS begins with the discovery of observable activities in hosts, which are given to the preprocessor, which is responsible to extract the features from the raw data. Then they are provided to the detection engine, which uses an anomaly model to encodes the detected anomaly.
This information are given to a classification engine, which based on classification algorithm is able to decide if the received event is really an anomaly or not. Finally, in positive case, this decision is used to raise alerts or to perform blocking actions. With IDS/IPS we can monitor several activities such as :
\begin{itemize}
\item reconnaissance (each type of activities that tries to recognize the services provided by our network)
\item patterns of specific commands in application sessions
\item content types with different fields of application protocols
\item network packet patterns between protected servers and the outside world
\item privilege escalation
\item attacks by legitimate users.
\end{itemize}
IDS/IPS can be either \textbf{host-based} or \textbf{network-based}, in which case it's labeled as HIDS/HIPS or NIDS/NIPS, respectively. In the former case, the monitoring and blocking activity is performed on a single host. HIDS/HIPS usually works in no promiscuous mode, and has the advantage that it provides better visibility into the behavior of individual applications running on that host. In the latter case, it is often located behind a router/switch (through a mirrored port) or firewall that provides the guarded entrance to a critical network area and work in promiscuous mode. At this location, the traffic is monitored and packet headers and payload are examined using the knowledge base in NIDS/NIPS. The advantage of this location is that a single NIDS/NIPS can protect many hosts as well as detect global patterns. Often a network-based IDS/IPS receives useful information from a series of sensors that are placed in different network areas such as DMZ, internal network, etc. The host-based application firewall perform the IPS function independently of the OS and block the entry of application level and web-based intrusions, much like network firewalls that blocks unwanted traffic. A network-based IPS blocks network level intrusions, such as DoS attacks, and may use anomaly detection to recognize threats based on their behavior. Combining network- and host-based IPSs provides the best protection against all types of intrusions. Regardless of the location of the IDS system, it should be capable of detecting a substantial percentage of intrusions with few false alarms. For example, if too few intrusions are detected (false negatives) there is really no security, while on the other hand too many false alarm (false positive) will eventually be ignored. 
In host-based IDS/IPS we have both advantages and weaknesses. They are capable of protecting mobile hosts from an attack when outside the protected internal network, and they can defend local attacks such as malware in removable devices. They also protect us against attacks from network and encrypted attacks in which the encrypted data stream terminates at the host being protected. They have the capability of detecting anomalies of host software execution (e.g. system call patterns). In contrary, if an attacker takes over a host, the HIDS/HIPS and NAC agent software can be compromised and disabled, and the audit logs are modified to hide the malware. In addition, HIDS/HIPS has only a local view of the attack, and a host-based anomaly detection has a high false alarm rate. The input to HIDS/HIPS are network packets, system logs, system events and hardware information. Combined signature- and behavior-based methods detect and block abnormal activity patterns and generate the system alarms and event reports.
In network-based IDS/IPS the incoming packets enter the \textbf{flow state table}, where state information is maintained. This maintenance of state information permits sensors to obtain the context for attack detection. The entire content of the data packet is inspected, and state information is captured and updated in real time. This state information provides the basis for layer 2-7 detection. Multiple token matches are utilized to capture attack signatures and behaviors that span packet boundaries or exist in an out-of-order packet stream. This process permits the system to determine if a state transition should be allowed or not, and thus detects or block malware, key loggers, etc. Then we have the \textbf{normalization function}, which involve both TCP and IP normalization, that are used to make comparisons with normal behavior. With the former one, invalid or suspect conditions are inspected (for example a SYN from server to client). Certain types of network attacks are blocked (for example insertion and evasion attacks). Insertion attacks occur when the inspection module accepts a packet that is rejected by the end host system, whereas evasion attacks occurs when the inspection module rejects a packet although the end host system accepts it. Segments that contains bad checksum, payload length, as well as TCP flags are discarded. TCP normalization is configured by assembling various TCP commands into a parameter map for filtering as a policy. The IP normalization process inspects packets using a configured parameter map for such things as general security checks, ICMP security checks, IP fragment reassembly, etc. The items used to configure the IP normalization parameter map are ToS, TTL, MTU, fragment reassembly, unicast reverse path and maximum number of fragments permitted. The final building block is constituted by \textbf{flow classification}, which decide the actions that the network-based IDS/IPS must take (e.g. modify firewall rules, alert, packet dropping/flow termination, traffic shaping). NIDS/NIPS may not detect encrypted traffic since some portions of data and some header information are encrypted. Malware are exploring this weakness of NIDS/NIPS by encrypting packets. In addition, NIDS/NIPS requires an intensive computation facility and may not have the computation power for countering mutated malware.\\The approaches to intrusion detection can generally be classified as :
\paragraph{Anomaly/Behavior-based} Anomaly-based detectors generate the normal behavior/pattern of the protected system, and deliver an anomaly alarm if the observed behavior at an instant doesn't conform to expected behavior. It's more prone to generating false positives due to the dynamic nature of networks, applications and exploit. Due to the difficulty of manually setting the profiles for complicated and dynamic traffic, anomaly-based detection should be applied at various levels of traffic aggregation, such as a single server in order to achieve the accurate protection. According to the type of protection, anomaly detection techniques can be classified into three main categories :
\begin{itemize}
\item \textbf{statistical-based} : the behavior of the system is represented from the captured network traffic activity and a profile representing its stochastic behavior is created. This profile is based on metrics such as the traffic rate, the number of packets for each protocol, the rate of connections, etc. This method uses the collected profiles that are related to the behavior of legitimate users, in statistical tests to determine if the behavior under detection is legitimate or not. During the anomaly detection process, one corresponding to the currently captured profile is compared with the previously trained statistical profile. As the network events occur, the current profile is determined and an anomaly score is estimated by comparing the two behaviors. Typically, the score indicates the degree of deviation for a specific event, and the IDS/IPS will flag the occurrence of an anomaly when the score surpasses a certain threshold. The threshold detection uses thresholds that are independent of users for examining the frequency of occurrence of events. In contrast, profile detection uses a profile of activity of each user/device to detect abnormal behavior. Statistical approaches have a number of advantages. First, they don't require prior knowledge about the normal activity of the target system; instead they have the ability to learn the expected behavior of the system from observations. Second, statistical methods can provide accurate notification of malicious activities occurring over long periods of time. However, the drawback include setting the values of the thresholds, parameters/metrics that is a difficult task, especially because the balance between false positives and false negatives is affected.
\item \textbf{knowledge-based} : it captures the normal behavior from available information, including expert knowledge, protocol specifications, network traffic instances, etc. The normal behavior is represented as a set of rules. Attributes and classes are identified from the training data or specifications. Then a set of classification rules, parameters or procedures are generated. The rules are used for detecting anomaly behaviors. Specification-based anomaly methods requires that the model is manually constructed by human experts in terms of a set of rules that describes the system behavior. This methods produce a low rate of false alarms, but are not as effective as other anomaly detection methods in detecting novel attacks, especially when it comes to network probing and DoS attacks. The most significant advantages of these methods are the low false alarm rate and the fact that they may detect zero-day and mutated attacks. The main drawback is that the development of high-quality rules is time-consuming and labor-intensive.
\item \textbf{machine learning-based} : Machine learning IDS/IPS schemes are based on the establishment of an explicit or implicit model that allows the patterns analyzed to be categorized. Machine learning is different from statistical-based methods because machine learning discovers the characteristics for building a model of behaviors. As more learning is performed, the model will become more accurate. The discovery and learning process is the advantage of machine learning. However, it requires a significant amount of computational resources.
\end{itemize}
\paragraph{Signature-based} Signature-based schemes capture defined signatures within the analyzed data in order to create a signature database corresponding to known attacks. It's efficient and accurate for signature-based detector to identify known attacks using a signature database. The signature content is often a string of characters that appears in the payload of packets as part of the attack. Once a new vulnerability is disclosed, signatures are developed by researchers to counter threats. Signature-based systems take a look at the payload and identify whether it contains a matched signature. While this detection scheme has a lower false positive rate, it may not detect zero-day and mutated attacks. Malware can be stealthy by embedding its communications into protocols that are likely to be present in normal network operations or incorporate polymorphism and metamorphism to avoid a fixed signature. Another example is the following, suppose that our signature-based schemes want to detect "USER root" and classify it as malicious activity. If we only have a scanning for every packets looking for the desired sequence, maybe one of the way to evade this technique is to split "USER root" in several packets, so that when the destination recombine them we would have "USER root". So we have to records previous packet's text to avoid this type of attack, but maybe the attacker can send packets out of order. We need to have a full reassembly, but also this technique is not enough, since the attacker can use TCP tricks so that certain packets are seen by NIDS but dropped by the receiving application (for example the attacker can manipulate checksum, TTL, fragmentation, etc). The big challenges to signature-based IDS/IPS are the size of signature database, and the processing time of packets against all entries in the signature database. These can make the IDS vulnerable to DoS attacks. Some IDS evasion tools flood signature-based IDSs with too many packets, thus making the IDS drop packets and fail detection.
Combined anomaly-based and signature-based IDS/IPS provides the best protection. The dominant technology in commercial systems is based on a set of rules, that in some way identify which is the security problem. This practice is very critical, because it's a way to encode an anomaly in a rule, you have to be able to synthesize the features that characterize an intrusion. A common tool that provide this detection scheme is \textbf{Snort}, but we will describe it in details in the next section. \\\\With behavior-based detection, the behavior is characterized by the state of the protected host/network. A baseline of normal behavior is developed, and then when an event falls outside this normal behavior pattern, it is flagged and logged. The profile of normal behavior consists of a comprehensive list of parameters and values for the target being monitored. During IDS/IPS installation, the administrator can select an appropriate profile based on the network zone's mission or service types. A profile template provided by the IDS/IPS vendor, is a collection of policy construction rules that the IDS/IPS uses to create the zone policies during the policy construction phase of the learning process. Based on the characteristics of the zone traffic, each policy template enable the IDS/IPS to produce a group of policies during the policy construction phase. Thus, the IDS/IPS uses these policies to monitor the zone traffic for anomalies that indicate an attack on the zone. Typically, these policies are configured in such a way to take actions against a particular traffic flow if this latter exceeds the policy thresholds. However, due to the ever-changing nature of network traffic, applications and exploits, false detection may occur. The self-learning capability supports learning the patterns of network usage and traffic patterns may take place during normal network operations. Consequently, adaptive profiles can reflect the normal network traffic pattern evolution, and thus avoid raising false alarms.\\\\Identifying a host that an attacker compromised by stealing a username and password is difficult to detect with a traditional IDS/IPS. The \textbf{Honeypot (HP)} technologies are decoy computer resources for the purpose of monitoring and logging the activities of entities that probe, attack or compromise them. The main assumption in implementing a HP, that appears to be part of a network but which is actually isolated and protected, is that only the attackers would be attracted to the resources provided by the HP. Therefore, any attempt to use those resources is expected to be malicious and should be monitored. HP logs the steps of actions performed by attackers, and can gain useful information that an IDS/IPS can't discover.\\\\So, the signature-based detection clearly indicates the detected attack method, while the behavioral-based indicate the attack type, the rule and the statistical profiles that was violated. But when we use the standard behavior-based alert, we really can't be sure on what actually happen. We have a variation in statistics measures but we can't really understand why this happen. The idea is that the signature-based can run by itself, doesn't need too much user activities. If you use the behavioral protection then we need some human interventions, because we have to see why there was the deviation, and understand what was going on. The best solution is try to combine both signature and behavioral protection, because the false negatives is a problem with signature-based scheme, because if you have an attack that is not in your signature database, then it won't be detected. On the other side, false positives is a problem with behavior-based scheme, when a legitimate action is classified as an attack due to the behavior analysis. A way to combine these two schemes is considering the detection correlation. Once we have an anomaly we can try to synthesize it, i.e. collect the elements that make an alert to be raised by the attack behavior, and try to determine what is the signature of this new attack. This could be a protection against zero-days attacks, since they try to lower the false positive rates and also reduce the response time to attacks.
\subsection{Distributed approach}
Organizations must defend their information infrastructures consisting of a distributed collection of hosts connected by a network and the defense can be enhanced through coordination and cooperation among the intrusion detection systems that exist throughout the network infrastructure. Such an approach is predicated on the existence of a distributed intrusion detection system, and extends the focus from a single unit to the entire information infrastructure, thus providing a more effective defense including agent-based coordination between the host and the NAC server. The aim of this distributed IDS is to improve the discovery of anomalies by reducing the rate of false positives and false negatives. Detection entities communicate through multicast, secure channels and correlate the different alerts emitted by local sensors. The severity of anomalies is evaluated through an accumulation of alerts scores. This correlation yields a more accurate picture of the network attacks that were either blocked or actually reached the internal network. In addition, the NAC correlates HIDS and NIDS in order to constantly monitor and block malicious activities. There are several issues that must be addressed in the design of a distributed IDS. For example, the distributed IDS may need to understand different audit record formats while one or more IDS in the network serves as the collection and analysis point for the data, which must be securely received and stored. The architecture can be either \textbf{centralized}, i.e. a single point, which is simpler but represents a bottleneck, or \textbf{decentralized}, i.e. multiple centers, that must be coordinated.
\subsection{Snort}
Snort is a open source NIDS/NIPS written by Martin Roesch, and is now developed by SourceFire, a company Roesch founded and currently serves there as CTO. This is the most widely deployed intrusion detection and prevention technology worldwide. It uses a rule-driven language that combines the benefits of signature and anomaly-based inspection methods. SNORT can be combined with other software such as SnortSnarf and OSSIM to provide a visual console. There is a large rule set for well known vulnerabilities, and community maintained Snort rule sets are evolving for emerging threats. Snort is logically divided into multiple components, that works together in order to detect particular attacks and to generate output in a required format from the the detection system. A Snort-based IDS consists of the following major components as stated by  Rafeeq Ur Rehman book \cite{snort} :
\begin{itemize}
\item \textbf{Packet decoder} : it takes the packets from different types of network interfaces and prepares them to be preprocessed or to be sent to the detection engine.
\item \textbf{Preprocessors} : they are components or plugins that can be used with Snort to arrange or modify data packets before the detection engine does some operation to find out if the packet is being used by an intruder. Some preprocessors also perform detection by finding anomalies in packet headers and generating alerts. They are also used for packet defragmentation, HTTP URI decoding, TCP stream reassemble and so on.
\item \textbf{Detection engine} : the detection engine is the most important part of Snort. Its responsibility is to detect if any intrusion activity exists in a packet. The detection engine employs Snort rules for this purpose. The rules are read into internal data structures or chains where they are matched against all packets. If a packet matches any rule, appropriate action is taken (e.g. logging the packet or generating alerts); otherwise the packet is dropped. It's also the time-critical part of Snort, depending how powerful your machine is and how many rules you have defined, it may take different amount of time to respond to different packets. 
\item \textbf{Logging and alerting system} : depending upon what the detection engine finds inside a packet, the packet may be used to log the activity or generate an alert. Logs are kept in simple text files, tcpdump-style files or some other form.
\item \textbf{Output modules} : these modules control the type of output generated by the logging and alerting system (e.g. sending messages to syslog facility, generating XML output, and so on).
\end{itemize}
All Snort rules have two logical parts : \textbf{rule header} and \textbf{rule options}. The rule header contains information about what action a rule takes (action, protocol, source address and port, direction, destination address and port). It also contains criteria for matching a rule against data packets. The options part contains additional criteria for matching a rule against data packets (brief text description, other fields to be matched such as TTL), typically enclosed by the parentheses. The biggest Snort challenges are : \textbf{misuse detection} to avoid well known intrusions, but the database size continues to grow (it spends around $80\%$ of time doing string matching) and \textbf{anomaly detection} to identify new attacks, but the detection probability is very low.
\subsection{Suricata}
Suricata comes as a revolution in order to overcome some of the Snort limitations. The main purpose of Suricata is to be high performance NIDS and also a Security Monitoring engine. It's a open source project and it wants to become a real time intrusion detection system, inline intrusion prevention system, network security monitoring and offline pcap preprocessing. The most critical difference is that Suricata makes
possible to parallelize the detection engine, leveraging hardware acceleration (like GPU acceleration). It has supports for many packet decoding and realize the IP reputation mechanism, i.e. every IP that you see in the network can have a reputation, according to the type of alert of the behavior that has in the network. An IP address can lose some kind of reputation, and when this value becomes lower than a certain threshold we need to take some actions (e.g. block the host with that IP address). Similar to Snort, it can be integrated with other solutions such as SIEMs and databases using YAML and JSON files as inputs and outputs. It incorporates the Lua scripting language to create more effective rules with respect to Snort.
\subsection{Fail2ban}
Fail2ban is a host intrusion prevention system that protects computer servers from brute-force attacks. The idea of Fail2ban is to having a mechanism that continuously scan the logs of a machine using some mechanism of detection and bans IP addresses of hosts that have too many failures within a specified time window. Fail2ban is based on the jail definition, which is one of the fail2ban-server thread that watches one or more log files, using one filter (regex) and can perform one or more actions. It's a client-server mechanism, multi-threaded, has a lot of predefined support such as sshd, apache, etc. and has the capability to autodetect the datetime format. It's a very lightweight and effective process, but its main disadvantage is the reaction time. In fact, it's a log parser, so it cannot do anything before something is written to the log file.
\subsection{Security Information and Event Management}
SIEM is a centralized place where we can receives data from several sources in the network and it tries to arranges together some of them in order to have an augmented/advanced picture of the situation in the network. SIEM is actually a really broad term, since it's a combination of tools, able to provide the following features : automatically collect and process information from distributed sources, store it in one centralized location, correlate between different events and produce alerts and reports based on this information. The features that we need to have in SIEM are the following :
\begin{itemize}
\item \textbf{Log management} : it means to collect and receive logs from different sources. Every nodes in your system generates these logs, usually related to events. In a well designed security system, all the logs generated by single nodes are usually sent and stored in a centralized database, i.e. source of information of our SIEM. The SIEM database operates several transformations of these logs, i.e. parse and normalize the data in order to have a standard view of the logs received by different sources. One of the task appreciated of the SIEM, is that they should be used to access the data in near real-time fashion and able to be used for data mining according to each of the element that we have in the system.
\item \textbf{IT regulatory compliance} : it's a way to be able to prove that you are doing your job well. This is done because you can expect to have external audit, i.e. somebody comes in your organization and checks if the things that you're doing are done in a good way. SIEMs are able to generate this kind of audit and also to validate the compliance, or to identify violations of compliance requirements imposed upon the organization. Since SIEMs can receive this kind of logs, they can generate reports, which are needed by businesses to provide evidence of self-auditing and to validate their level of compliance.
\item \textbf{Event correlation} : the idea is that the event correlation is like one of the most critical elements of the SIEM, because it makes possible to combine together different sources in order to provide a picture of the events. We need to remember that the correlation engine try to correlate together different events that may be happens in different parts of the network so that they can be merged together and used to provide a better picture of the system. Providing this event correlation is very complex, because you have to be able to known your network and tell the system how all the sources of the data are related together. It's not so easy make this mechanism effective.
\item \textbf{Active response} : there are some automated procedures that can be enabled for some given security events such as automatic response and manual response. It's similar to what we have in IDS, but it can be done at a higher level, only after having more evidences that something happens. Instead of reveal a single element, we can combine together several elements and only then take the decision. This is not really common to have, because actions can be disrupted in the system.
\item \textbf{Endpoint security} : it means that from the SIEM we can monitor the health security of a system. It's a way to be able to manage from a remote location the different endpoints that we have in our system. We can also force push down and install updates or may be realize some kind of manipulations of the system.
\end{itemize}
As we could imagine, logs are the main element of the SIEM. The idea is that according to the type of raw data that we have, we can extract the relevant information. When dealing with the logs we need to answer to several questions : 
\begin{enumerate}
\item \textbf{How long must you retain the logs ?} this first question brings up questions regarding data retention and data destruction. Industry regulations or laws may require you to retain certain types of data for a given period of time (data retention). You may also have legal and functional drivers that dictate how you dispose of information after a given period of time (data destruction).
\item \textbf{How much log information will you be required to retain?} you will want to define what a reasonable amount of data storage is for your environment, and in combination with your data retention and data destruction requirements, use that information to decide what kind of data you need to retain.
\item \textbf{What kind of information system logs are you required to retain ?} event information comes in many sizes and shapes, and a critical key to both the issue of log management and also the ultimate usefulness of your SIEM solution is in choosing what kind of information you will retain.
\end{enumerate}
At this point, based on what we have said, we can refers to the \textbf{SIEM stack}, which is constituted by :
\begin{itemize}
\item \textbf{Event layer} : it's where we collect logs and other event messages from systems on our network.
\item \textbf{Normalization layer} : it's where we convert related messages that are formatted differently to a common syntax.
\item \textbf{Correlation layer} : it's where events are related to each other to create incidents.
\item \textbf{Reporting layer} : it's where output is created and/or actions taken based on the processing of the events that have been input into the SIEM system.
\end{itemize}
When we have a SIEM one of the critical part is how we interface with it, i.e. how we expect to interact with SIEM. We expect to have dashboards and maps. The idea is to have a pull of information, and we can extract useful information from the dashboard. Usually, it's a graphical and organized representations of alerts, event data and statistical information. In this way, it try to help administrators to see patterns, understand trends and identify unusual activity. Other operational interface are the alerts, we can receive a push of information, which are related to critical events that occurs in the network. The SIEM main disadvantage is that it's complex. In fact, a survey conducted in 2013 revealed that it's very complex to manage the complexity of the product when using SIEM, the previous problem is due by lack of trained personnel to manage the product and lack of integration with other products. Other points related to this survey are that some of them would prefer to replace their existing SIEM solution for better cost savings, others spends too much on the system with respect to their current SIEM implementation, and others require two or more full-time employees to manage the chosen solution.
\section{Penetration testing and vulnerability assessment}
The two activities are surely related but are different. In fact, the distinction between vulnerability assessment and penetration testing is that the first one is a process to identifying and quantifying the vulnerabilities that we have in a system. The latter one is something similar but in some way it should be a process of evaluating a series of attacks that we can have in our system. The idea is that we don't really see a system as a whitebox, in the sense that we known how it works, but we see it as a whole, without knowing the details. Typically the vulnerability assessment is something that is done by the administrative entity of the system, while the penetration testing usually is an activity that is given to do to a third party, somebody not really involved in the system that tries to simulate the activity that an attacker could do to the system. Lets start talking about penetration testing. It's not really a well established procedure, but from a general point of view it's constituted by the following activities :
\begin{itemize}
\item \textbf{Planning} : it's a way when we start understanding the perimeter of our activity, where it can go and how far it should start. We need to agree also the type of activities that we can perform in the network, like DoS attacks or social engineering. This phase includes the details that we have to agree with penetration testing client such how long the activity should take ? when we are suppose to do the activity ? when we are allowed to perform some kind of activities in the system ? 
\item \textbf{Non-intrusive target search (intelligence gathering)} : in this phase there will be a collection of information without really interacting with the system. It's a procedure to just try to get as much as possible data from the system, without direct interaction with it. It's really a phase that we can performs with several tools such as nslookup, whois, google hacking (collect information about the target system through the web), etc.
\item \textbf{Intrusive target search} : it's a more active part, where we really interacts with the system and try to understand the type of services that the system provides. This procedure will involve some kind of scanning activities. We need to have an explicit agreement that we have the permissions to perform this activity, because otherwise it's illegal. The aim of this activity is to finds the systems in the network, discover the open ports, discover the services that are running in the ports and the enumeration, which try to see the versioning of the services (like OS, software version, level of patching, etc). These activities are usually performed by network scanners.
\item \textbf{Data analysis} : it's the most extensive part. When we perform a scan we can really obtain a huge amount of data. The idea is to be able to understand the data that we have collected and in some way try to find the interesting parts. In this phase we can determine if there is some missing elements and thus repeating the previous steps to fill the gap, because may be we can realize that there is an additional test that we are supposed to perform in the system.
\item \textbf{Threat modeling and Exploitation} : it's the most disruptive part, when we try to find out the weaknesses of the system and to exploit them in order to obtain some extra information. This is probably the most time consuming part, where we have to figure out if we can and how exploit the different services. Usually is a process based on the user experience. In addition to this manual activity, it can also perform some automated tools like nessus, OpenVAS, Burp suite, etc. Exploitation is a really broad term. It's a way to exploit something to obtain some information or may be directly violate the security policy of the system.
\item \textbf{Reporting} : when we fill a series of documents that will be given to our customer. Usually the report should include some suggestion on how to tackle with issues that we found in our activity. 
\item \textbf{Repeat} : after this reporting phase, there will be an active part from the point of view of the administration, like acting to take some countermeasures to evolve their network with respect to the suggestions that we gave them, so that we can repeat the penetration testing activity in order to see if some of the issues that we found are in some way fixed and if these fixes were opening some additional issues.
\end{itemize}
\subsection{Types of security test}
Lets introduce some terminology, a way to represent the possible testing we can perform on a network :
\begin{itemize}
\item \textbf{Vulnerability scanning} : they are automated checks for known vulnerabilities against a system or systems in a network.
\item \textbf{Security scanning} : they are vulnerability scans which include manual false positive verification, network weakness identification and customized professional analysis.
\item \textbf{Penetration testing} : it's a goal-oriented project which simulates an attack from a malicious hacker. It includes gaining privileged access by pre-conditional means.
\item \textbf{Risk assessment} : it involves an interview and mid-level research. With respect to the other activities is less technical, because usually requires the analysis of the procedures of an organization.
\item \textbf{Ethical hacking} : it's a penetration test of which the goal is to discover security flaws in the system and the network within the predetermined project time limit.
\item \textbf{Security testing} : it's a project-oriented risk assessment of systems and networks through the application of professional analysis on a security scan where penetration is often used to confirm false positives and false negatives as project time allows.
\item \textbf{Security auditing} : it's a privileged security inspection of the OS and applications of a system or systems within a network or networks.
\end{itemize}
The vulnerability and security scanning requires low time and low costs, the penetration testing and risk assessment requires medium time and medium costs, since they are more intrusive tasks that typically requires to be performed by a third party, and ethical hacking, security testing and auditing requires high time and high costs, since these activities requires explicit rights to be performed.\\\\Another evaluation of the different engaging possibilities that we have in penetration testing depends from the knowledge of the analyst and the knowledge of the owner about who is performing the testing. We can summarize the different degrees, as the OSSTMM manual states \cite{osstmm}, of knowledge in the following way :
\begin{itemize}
\item \textbf{Double blind} : the analyst engages the target with no prior knowledge of its defenses, assets or channels. The target is not notified in advance of the scope of the audit, the channels tested or the test vectors. A double blind audit tests the skill of the analyst and preparedness of the target to unknown variables of agitation. The breadth and depth of any blind audit can only be as vast as the analyst's applicable knowledge and efficiency allows. This is also known as a \textbf{black box test} or \textbf{penetration test}.
\item \textbf{Blind} : the analyst engages the target with no prior knowledge of its defenses, assets or channels. The target is prepared for the audit, knowing in advance all the details of the audit. A blind audit, as the double blind one, tests the skills of the analyst. The breadth and depth of a blind audit can only be as vast as the Analyst’s applicable knowledge and efficiency allows.
\item \textbf{Double gray box} : the analyst engages the target with limited knowledge of its defenses and assets and full knowledge of channels. The target is notified in advance of the scope and time frame of the audit but not the channels tested or the test vectors. A double gray box tests the skill of the analyst and the target's preparedness to unknown variables of agitation. The breadth and depth depends upon the quality of the information provided to the analyst and the target before the test as well as the analyst's applicable knowledge.
\item \textbf{Gray box} : the analyst engages the target with limited knowledge of its defenses and assets and full knowledge of the channels. The target is prepared for the audit, knowing in advance all the details of the audit. The breadth and depth depends upon the quality of the information provided to the analyst before the test as well as the analyst's skills.
\item \textbf{Reversal} : the analyst engages the target with full knowledge of its processes and operational security, but the target knows nothing of what, how or when the analyst will be testing. The true nature of this audit is to audit the preparedness of the target to unknown variables and vectors of agitation. The breadth and depth depends upon the quality of information provided to the analyst and the analyst's skills.
\item \textbf{Tandem} : the analyst and the target are prepared for the audit, both knowing in advance all the details of the audit. A tandem audit tests the protection and controls of the target. However, it cannot test the preparedness of the target to unknown variables of agitation. The true nature of the test is thoroughness as the analyst does have full view of all tests and their responses. The breadth and depth depends upon the quality of information provided to the analyst before the test as well as the analyst's knowledge.
\end{itemize}
\subsection{Vulnerability assessment}
A \textbf{vulnerability} is a weakness which can be exploited by a threat actor, such as an attacker, to perform unauthorized actions within a computer system. Instead a \textbf{vulnerability scanner} is a tool that discovers and enumerates the vulnerabilities on systems or applications. They are useful because can be run without so much effort, and can repeat a checks for a lot of problems very quickly. They can also help to identify rogue machines or to provide a collection of the machines in the network and also to keep them updated. It also give us a generic disk level since it evaluates each system and provide us a picture of the type of issues that we can have. It can also explain why an item is a risk, because usually it's built starting from known vulnerabilities. The direct consequence of this is that it provides us detailed information on how to apply remediation, because usually a vulnerability is disclosed only after there is a patch for that. We have different types of vulnerability scanners such as :
\begin{itemize}
\item \textbf{Agent vs agent-less} : agents are monitoring software running on each end node and communicate with a central scanner. They are highly customizable and provides us very precise information. Agent-less means that we only perform the activity interacting from the outside with a system.
\item \textbf{Passive vs active} : passive scanners only monitor the network and are completely not intrusive (for example a NIDS). They can complement active scanners. Instead, active scanners real-time monitor of new entities in the network or configuration changes.
\end{itemize}
The main vulnerability scanners disadvantages is that they are slower than simpler port scanners, because they in addition to perform enumeration they also try to interact with the systems, try to understand if there are vulnerabilities, and figure out if they can be exploited. Second, some scanning can disrupt operations (for example performing a DDoS testing). Another critical point is that they generate a huge amount of false positives, requiring a human judgment. Typically vulnerability scanners are based on a vulnerability database, and thus it must be updated frequently. The security resources are often decentralized, i.e. it means we have several different position where the data are. Another really interesting point that we have to focus on when leveraging a vulnerability scanner, after we found a vulnerability, is that how can we fix the vulnerability ? If we can't fix it, we should ignore or accept it ?\\\\A vulnerability scanner follow a very straightforward procedure :
\begin{enumerate}
\item \textbf{Discovery} : it means check if the target is there and which kind of target are there.
\item \textbf{Port scans} : it tries to see which are the ports that are listening.
\item \textbf{Service detection} : it guess the type of service that is answering and expecting data on a given port. This can be performed starting from well known ports, but it can be also performed with trial and error technique.
\item \textbf{OS detection} : it tries to understand by guessing the operating system where the system is running on.
\item \textbf{Vulnerability assessment} : it tries to see if there is a vulnerability on the system and tries to determine if this is a known one and to see if this is still there in the system.
\end{enumerate}
Typically a well known vulnerability is identified by \textbf{Common Vulnerabilities and Exposures (CVE) ID}. A CVE is a well known database maintained by MITRE Corporation, which stores publicly known information security vulnerabilities. A CVE entry also called CVE ID provide a reference point for data exchange so that cybersecurity products and services can speak with each other. When you discover a new vulnerability you should send a report to the MITRE Corporation, which includes all the details of the vulnerability discovered, and there will be a period of time with which this organization will explore further details about it. Then the vulnerability is analyzed by NIST, which using the \textbf{Common Vulnerability Scoring System (CVSS)} provide both a severity score and 8 commonly adopted security attributes of the vulnerability.\\\\Some additional methodology that can be used in the vulnerability scanner are : manual attempts and permutation (e.g. we can give some of the inputs that are supposed to be given to the service and can be altered in random way), manual version probe, custom protocol-specific attacks and automated vulnerability scanners. Once we have a scan of the system, we obtain a spreadsheet with a list of the critical findings of the scanning. It can take a lot of time to go through the list of the data, so we can't really use the report as you get as a report for the customer. Thus we have to explore the report obtained from the scanning and try to find what can be really an issue. In other words we have to perform a prioritization task, understanding which are the most critical vulnerabilities found via the scanning. After that, we need to turn them into \textbf{actionable information}, i.e. they should be useful for the customer. These information can be suggestions or how to fix the vulnerabilities.
\begin{appendices}
\section{OpenVPN}
OpenVPN is a open-source software that implements VPN techniques to create secure point-to-point or site-to-site connections in routed or bridged configurations and remote access facilities. It uses a custom security protocol that utilizes SSL/TLS for key exchange. It's capable of traversing NATs and firewall. It allows peers to authenticate each other using pre-shared secret keys, certificates or username/password. It uses the OpenSSL library to provide encryption of both data and control channels. OpenVPN can run over UDP or TCP transports, multiplexing created SSL tunnels on a single UDP/TCP port. It offers two types of interfaces for networking via the Universal TUN/TAP driver. \textbf{TAP} (network tap) operates much like TUN, however instead of only being able to write and receive layer $3$ packets to/from the file descriptor it can do so with raw ethernet packets. Instead, a \textbf{TUN} (tunnel) devices operate at layer $3$, meaning the data (packets) you will receive from the file descriptor will be IP based. Data written back to the device must also be in the form of an IP packet.
\section{OpenVAS}
Open Vulnerability Assessment System is an open source framework to analyze and manage vulnerabilities. It automates over $50000$ different \textbf{Network Vulnerability Test (NVT)}, which are arranged in feeds. It's updated frequently, prioritizing the most critical vulnerabilities. The OpenVAS main components are :
\begin{itemize}
\item \textbf{Scanner} : it scan and receive signature updates of NVTs.
\item \textbf{Manager} : it manages the scanning rules, consolidates reports and stores settings and history based sqlite.
\item \textbf{Clients} : it's a command line and web GUI.
\end{itemize}
The typical procedure for scanning start with the creation of a new target, in which we specify the hosts and the ports to be scanned. Then we create a task, in which we specify the vulnerability assessment parameters such QoD, scan type, etc. Finally, the task is executed and once finished it will report useful information about the vulnerabilities eventually discovered.
\end{appendices}
\bibliographystyle{plainnat}
\bibliography{references}
\end{document}